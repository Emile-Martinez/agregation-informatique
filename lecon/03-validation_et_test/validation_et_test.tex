\dev{Emile Martinez, Malory Marin}{}{}{}

\section{Introduction}

\subsection{Qu'est-ce qu'un test}

Tester est un anglicisme pour le mot français essayer (ou éprouver). Me soumettant à la folie anglomane ambiante je garderai ce mot, soucieux de ma cohérence avec le monde extérieur.

\begin{definition}
	\textbf{Tester} un programme consiste à essayer d'y trouver des erreurs
\end{definition}

\begin{rem}
	On ne cherche pas ici à prouver directement que le programme est correct, mais à prouver qu'on n'arrive pas à se rendre compte qu'il est incorrect.
\end{rem}

\subsection{Données de tests}

\begin{definition}
	\textbf{Une donnée de test} est un couple (valeur d'entrée, valeur de sortie), où à l'évidence, le deuxième élément représente la valeur de sortie quand la fonction est appelée sur le premier élément.
\end{definition}

\begin{definition}
	\textbf{Un jeu de données de test} (ou jeu de tests) est alors un ensemble de tels couples, permettant de vérifier la validité du programme sur certaines entrées.
\end{definition}

\begin{rem}
	Certaines sorties (attendues) peuvent être des erreurs.
\end{rem}

\begin{example}
	Un jeu de tests pour une fonction calculant le pgcd peut être \\$\left\{ \big((1,2), 1\big), \big((-3, 6), 3\big), \big((0, 0), 0\big), \big((2, 2.45), \text{Erreur de type}\big) \right\}$
\end{example}

\subsection{Types de tests}

Il existe deux types de tests : 
\begin{itemize}
	\item Les tests en boites noires : On ne connait pas le code de la fonction, on peut simplement l'appeler. 
	\item Les tests en boites blanches : On connait le code et on génère un jeu de test en fonction. 
\end{itemize}

\section{Tests en boîtes noires}

\subsection{Caractéristiques}

Pour un test en boîtes noires, comme on ne connait pas le code, il faut tester beaucoup de données. Idéalement, toutes, mais cela se trouve souvent impossible.

\begin{example}
	On peut tester toutes les valeurs d'une fonction qui implémente une fonction booléenne mais pas celles de notre fonction calculant le pgcd de deux nombres
\end{example}

Viennent alors deux problèmes : Générer suffisament de données d'entrées et effectuer le test suiffisament rapidement.

\subsection{Générer des données d'entrée}

Dans de nombreux cas, ne pouvant pas essayer toutes les données d'entrées, on va devoir faire des choix.

\begin{idee}
	La première approche consisterait à générer des valeurs aléatoires dans un domaine, et espérer en prendre suffisamment pour que cela fonctionne.
\end{idee}

\begin{principe}
	Une approche plus maline, à partitionner le domaine, puis à appliquer l'approche naïve sur chaque domaine.
\end{principe}

\begin{example}
	Pour le calcul de pgcd($a$,$b$), on peut parititionner le domaine d'entrées en comparant $a$, $b$, et $0$, (avec donc 6 domaines : $a <= 0 <= b$, $0 <= a <= b$, $0 <= b <= a$, $a <= b <= 0$, $b <= a <= 0$, $b <= 0 <= a$)
\end{example}

\begin{rem}
	On se contente souvent de prendre un seul test par classe.
\end{rem}

\begin{rem}
	Le choix du partitionnement est arbitraire et doit donc être fait suivant la manière d'approcher le problème
\end{rem}


Une fois cela fait, il est très commun que les erreurs puissent venir des cas limites. 

\begin{principe}
	On essaye alors de se placer au limites des domaines, et de vérifier spécifiquement ces cas là.
\end{principe}

\begin{example}
	Pour l'exemple précédent, on testerai les cas d'égalité : $0 <= a = b$, $a = 0 = b$, $a  = 0 <= b$ etc... (en effet par exemple, le cas 0, 0 est différent des autres, la valeur pouvant être $+\infty$ ou $0$ suivant les définitions).
\end{example}

\subsection{Utiliser les caleurs de sorties efficacement}

Néanmoins, maintenant que l'on a les valeurs d'entrées, il faut pour avoir notre jeu de tests avoir également les valeurs de sorties.

\begin{example}
	si l'on veut générer la sortie du pgcd sur des entrées que l'on a pris au hasard, il faut connaitre déjà le pgcd. Quel intérêt d'avoir notre fonction alors si on a déjà une fonction qui le fait
\end{example}

\noindent On a alors plusieurs méthodes : \begin{enumerate}
	\item Générer un jeu de tests à la main
	\item Utiliser un programme moins performant mais que l'on sait correct.
	\begin{example}
		Si on calcule le pgcd par soustraction successives, on peut tester en calculant le pgcd en testant tous les nombres inférieurs à $a$ et a garder le plus grand qui divise $a$ et $b$.
	\end{example}
	\item Ne pas calculer la réponse mais simplement vérifier que la réponse fourni est correcte.
	\begin{example}
		Si on a un programme qui nous donne la décomposition en facteurs premiers d'un nombre, il nous suffit de tester la primalité de chaque sortie et de vérifier que leur produit fait l'entrée.
	\end{example}
\end{enumerate}

\begin{example}
	Si on a un algorithme performant effectuant le produit de matrice, on peut : \begin{enumerate}
		\item Créer à la main quelque petite matrice et faire leur produit pour vérifier que tout fonctionne
		\item Comparer avec l'algorithme naif du calcul de produit de matrice
		\item Vérifier de manière probabiliste que le résultat est bien le bon.\\
		\textbf{Developpement \ref{D?}} Vérification probabiliste du produit de matrice.
	\end{enumerate}
\end{example}

\section{Tests en boite blanche}

\subsection{Graphe de flot de contrôle}

Pour un test en boîte blanches, on connaît le code, et on va vouloir générer des données d'entrées en fonction de ce code là.\\

Pour cela, on extrait du code le graphe de flot de contrôle.

\begin{definition}
	Le graphe de flot de contrôle est un graphe où chaque boîte contient des lignes de codes, les boites sont reliés si on peut exécuter l'une puis l'autre.
\end{definition}

\begin{example}
	\label{3-1}
	On prend l'exemple du pgcd pour $a, b \in \N^*$\\
	
	\noindent \begin{minipage}{0.5\linewidth}
		\begin{algorithm}[H]
			\Tq{$a \neq b$}{
				\eSi{$a < b$}{
					$b \gets b - a$
				}{
					$a \gets a - b$
				}
			}
			\Retour{$a$}
		\end{algorithm}
	\end{minipage}\begin{minipage}{0.5\linewidth}
		\begin{tikzpicture}[->, node distance=2cm]
			\node[elliptic state, scale = 1] (q0) {Début};
			\node[elliptic state, scale = 1, below right of = q0] (q1) {$a \neq b$};
			\node[elliptic state, scale = 1, above right of = q1] (q2) {Fin};
			\node[elliptic state, scale = 1, below left = 0.8cm and 1.5cm of q1] (q3) {$b \gets b - a$};
			\node[elliptic state, below right = 0.8cm and 1.5cm of q1] (q4) {$a \gets a - b$};
			\node[elliptic state, below right = 0.8cm and 1.5cm of q3] (q5) {$a < b$};
			
			\draw (q0) edge[] node{} (q1);
			\draw (q1) edge[left] node{Oui} (q2);
			\draw (q1) edge[left] node{Non} (q5);
			\draw (q3) edge[] node{} (q1);
			\draw (q4) edge[] node{} (q1);
			\draw (q5) edge[left] node{Oui} (q3);
			\draw (q5) edge[right] node{Non} (q4);
		\end{tikzpicture}
	\end{minipage}
\end{example}

\subsection{Utilisation du graphe}

On essaye alors de générer un jeu de données qui parcourt une bonne partie du graphe.

Par exemple, un jeu couvrant : \begin{itemize}
	\item Tous les nœuds (On veut un jeu de tests tel que tous les tests pris ensemble, chaque nœud du graphe est parcouru au moins une fois).
	\item Tous les arcs
	\item Tous les chemins
\end{itemize}

\begin{com}
	Si on veut aller plus loin, on peut aussi rajouter toutes les conditions décisions, toutes les p utilisations, etc... On peut alors rajouter un exercice proposant de montrer la hiérarchie entre ces tests.
\end{com}

\begin{example}
	Sur l'exemple \label{3-1}, $\left\{ \big( (1,1), 1\big), \big((1,3), 3)\big)\right\} $ ne couvre pas tous les nœuds quand $\left\{ \big((1,3), 3), \big( (3, 1), 3 \big) \right\}$ couvre tous les nœuds et tous les arcs mais pas tous les chemins.
\end{example}

\begin{rem}
	Quand il y a une boucle, tous les chemins peut-être un critère infini. On peut alors se limiter aux chemins d'une certaine taille.
\end{rem}

\begin{rem}
	Parfois les critères sont insatisfiables.
\end{rem}

\begin{rem}
	Aucun de ces critères ne garantissent la validité d'un algorithme. Elles permettent simplement de vérifier que notre jeu de tests n'est pas trop lacunaire.
\end{rem}

\noindent \textbf{Developpement \ref{D?}} Intérets et insuffisances de ces critères

\subsection{Test exhaustif de condition}

\begin{idee}
	Une autre approche consiste à avoir un jeu qui satisfait ou invalide toutes les conditions de toutes les manières possibles.
\end{idee}

\begin{example}
	On voit l'utilité sur l'exemple suivant :\\
	\begin{minipage}{0.5\linewidth}
		\begin{lstlisting}[style=CStyle]
int max(int a, int b){
	if(a > b || a == 500){
        return a;
    else
        return b;
}
\end{lstlisting}
	\end{minipage}
	\begin{minipage}{0.5\linewidth}
		Pour détecter le problème (que max ne calcule pas le $\max$), il faut des tests où on mets à vrai le premier if à cause de $a == 500$, donc des tests où $a$ vaut $500$.
	\end{minipage}
\end{example}

\section{Pratiques pour éviter d'avoir à déboguer}


Dans la pratique, de bonne pratique de code sont très efficace pour éviter de passer trop de temps à debugguer son code.

\begin{itemize}[label=$\star$]
	\item Compiler avec -Wall (activant tous les warnings, donnant beaucoup de bugs stupides)

	\item Respecter la ponctuation et éventuellement utiliser un linter (de manière à rendre lisible le code par d'autres personnes)

	\item Faire de la programmation défensive en utilisant assert par exemple
\end{itemize}

