\dev{Emile Martinez}{}{}{}


\begin{com}
	Ici correction est pris majoritairement dans le sens «est correct» et non dans le sens de «corriger», à savoir rectifier car c'est souvent dans ce sens là qu'on parle de correction de programme.
\end{com}

\begin{com}
	On se concentre ici surtout sur les outils théoriques, en estimant que le terme «méthode» de l'intitulé de la leçon laisse penser que c'est sur ceux-là qu'il faut se concentrer
\end{com}

La conjecture de Syracuse est un problème ouvert de Mathématiques : \\
La suite $u$ définie par : $\left\{ \begin{array}{l}
	u_0 = a \in \N^*\\
	u_{n+1} = \left\{\begin{array}{ll}
		\dfrac{u_n}{2} & \text{si } u_n \text{ est pair}\\
		3u_n+1 & \text{sinon}
	\end{array}\right.
\end{array}\right.$ finie-t-elle toujours par le cycle $1,2,4$ ? (toujours = pour tout $a\in \N^*$)\\

Cela revient à savoir si l'algorithme \\
\begin{algorithm}[H]
	\caption{Syracuse(a)}
	$u \gets a$\\
	\Tq{$u$ est une nouvelle valeur}{
		\eSi{$u$ est pair}
		{
			$u\gets \dfrac{u}{2}$
		}{
			$u \gets 3u+1$
		}
	}
	\Retour{$u$}
\end{algorithm}
renvoie toujours 1, 2 ou 4.

\begin{com}
	C'est un exemple fleuve qui nous permettra d'illustrer toutes les notions autour de la correction et leur non trivialité
\end{com}
\section{Terminaison}
Une première question est de savoir si \texttt{Syracuse} finit (ne boucle pas à l'infini) sur toute entrée.

\begin{definition}
	Prouver la terminaison d'un algorithme revient à prouver que sur toute entrée il termine
\end{definition}

\begin{rem}
	On se limite parfois aux entrées valides (pour Syracuse, $a \in \N^*$ par exemple)
\end{rem}

\begin{example}
	\enspace \\ \\
	\begin{minipage}{0.4\linewidth}
		\begin{lstlisting}
Tant que a > 0:
    a = a - 1
		\end{lstlisting}
	\end{minipage} \quad \begin{minipage}{0.5\linewidth}
		Termine sur tout entrée si on n'autorise pas $a$ à valoir $+\infty$, sinon ne termine pas sur toute entreé.
	\end{minipage}
\end{example}

\begin{definition}
	Un \textbf{variant} est une fonction des variables, à valeurs dans $\N$ qui décroit strcitement :
	\begin{itemize}
		\item à chaque passage dans la boucles pour les algorithmes itératifs
		\item à chaque appel récursif pour les algorithmes récursifs.
	\end{itemize}
\end{definition}

\begin{example} \enspace \\ \\ \label{1-2}
	\begin{minipage}{0.5 \linewidth}
		\begin{algorithm}[H]
			\caption{pgcd($a$,$b$)}
			\Tq{$\min(a,b) > 0$}{
				\eSi{$a<b$}
				{
					$b\gets b-a$
				}
				{
					$a\gets a - b$
				}
			}
			\Retour{$\max(a,b)$}
		\end{algorithm}
	\end{minipage} \quad \begin{minipage}{0.4 \linewidth}
		La fonction \texttt{pgcd(a,b)} qui calcule le pgcd de $a$ et $b$ pour $a, b \in \N$ admet comme variant $a+b$ (qui est en effet toujours positif, et décroit à chaque fois)
	\end{minipage}
\end{example}

\begin{com}
	Nous réutiliserons plusieurs fois cet exemple du pgcd
\end{com}

\begin{proposition}
	\label{1-1}
	Si une boucle a un variant de voucle, alors elle s'éxecute un nombre fini de fois. De même pour un algorithme récursif.
\end{proposition}

\begin{example}
	pgcd($a$,$b$) termine pout tout $(a,b)\in \left(\N^*\right)^2$
\end{example}

\begin{rem}
	Si un algorithme termine sur toute entrée, il existe toujours un variant, mais il peut être difficile à trouver (ou à prouver)
\end{rem}

\begin{com}
	Il suffit de prendre comme variant le nombre d'étapes de calcul restantes en focntion de l'état de la mémoire
\end{com}

\begin{example} \enspace \\
	\begin{minipage}{0.4\linewidth}
		On définit $ack(n,m)$ pour $n,m\in \N$ par \\ $\left\{\begin{array}{l}
			ack(0,m) = m+1\\
			ack(n, 0) = ack(n-1, 1)\\
			ack(n,m) = ack(n-1, ack(n, m-1))
		\end{array}\right.$
	\end{minipage}\quad
	\begin{minipage}{0.55\linewidth}
		\begin{algorithm}[H]
			\caption{ack($n$,$m$)}
			\eSi{$n=0$}{
				\Retour{$m+1$}
			}{
				\eSi{$m = 0$}
				{
					\Retour{ack($n-1$, $m$)}
				}
				{
					\Retour{ack($n-1$, ack($n$, $m-1$))}
				}}
		\end{algorithm}
	\end{minipage} \\Il n'est pas immédiat que $ack$ termine.
\end{example}

\begin{definition}
	On dit qu'un ordre est un \textbf{ordre bien fondé} si toute suite décroissante est stationnaire
\end{definition}

\begin{example}
	$\N$ avec l'ordre naturel est bien fondé 
\end{example}

\begin{proposition}
	Les ordres produit et lexicographiques d'ordre bien fondés sont bien fondés
\end{proposition}

\begin{example}
	Un ordre total sur un ensemble fini est bien fondé, donc l'ordre alphabétique est bien fondé (c'est un ordre lexicographique)
\end{example}

\begin{definition}
	On étend alors la définition du variant aux fonctions à valeurs dans un bon ordre.
\end{definition}

\begin{proposition}
	La propriété \ref{1-1} reste valide avec notre définition étendue du variant
\end{proposition}

\begin{example}
	Pour ack, $(n,m)$ est un variant dans $\N^2$ avec l'ordre lexicographique, donc ack termine
\end{example}

\section{Correction partielle}

Une autre question pour \texttt{Syracuse} est de savoir si on peut tomber sur un autre cycle que 1,2,4 et donc renvoyer autre chose que 1,2 ou 4.

\begin{definition}
	On appelle \textbf{spécification} d'un algorithme deux propriétés $P_1$ sur les entrées (pré-condition) et $P_2$ sur les sorties (post-condition)
\end{definition}

\begin{example}
	Pour Syracuse, $P_1 : «a\in \N»$ et $P_2 : «\text{Syracuse}(a) \in \{1,2,4\}»$
\end{example}

\begin{definition}
	On dit qu'un algorithme est \textbf{partiellement correct} si pour toute entrée vérifiant la pré-condition, si l'algorithme termine, la sortie vérifie la post-condition.
\end{definition}

\begin{example}
	L'algorithme de l'exemple \ref{1-2} est partiellement correct si la pré-condition est «$a\in \N, b \in \N$» et la post-condition «\texttt{pgcd(a,b)} renvoie le PGCD de a et de b» \label{1-3}
\end{example}

\subsection{Correction partielle des algorithmes impératifs}

Pour prouver la correction partielle des langages impératifs, on utilise un invariant de boucle.

\begin{definition}
	Un invariant de boucle est une propriété qui est vrai avant la boucle, et si elle est vraie quand on commence un tour de boucle, alors elle l'est quand on le finit.
\end{definition}

\begin{proposition}
	Si un invariant de boucle est valide, alors il est vrai après la boucle, et la condition d'arrêt de la boucle est fausse.
\end{proposition}

\begin{example}
	Pour pgcd, «$PGCD(a,b) = PGCD(a_0, b_0)$» où $a_0$ et $b_0$ sont les valeurs initiales de \texttt{a} et \texttt{b}, est un invariant valide.
	
	A la fin de l'exécution, on a donc $\min(a,b) = 0$ et $PGCD(a,b) = PGCD(a_0, b_0) = PGCD(\min(a,b), \max(a,b)) = PGCD(0, \max(a,b)) = \max(a,b)$. D'où l'assertion de l'exemple \ref{1-3}
\end{example}

\subsection{Correction partielle des algorithmes récursifs}

\begin{com}
	On ne présente souvent pas la correction partielle des algorithmes récursifs, en se contentant de la correction totale directement. Néanmoins on présente quand cela, car cela illustre très bien la manière de raisonner quand on veut coder en récursif (et que tout est vrai)
\end{com}

\begin{principe}
	Pour prouver la correction partielle d'un algorithme récursif, on vérifie que si \begin{itemize}[label=$\bullet$]
		\item la pré-condition est vérifiée
	\end{itemize}, alors
	\begin{itemize}[label=$\bullet$] 
		\item on ne fait que des appels récursifs où les arguments vérifient la pré-condition
		\item Si la post-condition des appels récursifs est vérifiée, alors celle de notre appel est vérifiée
	\end{itemize}
	\label{1-5}
\end{principe}

\begin{theorem}
	Si le principe \ref{1-5} est respecté, alors l'algorithme est partiellement correct.
\end{theorem}

\begin{example}
	\label{1-7}
	exp($a$,$n$) pour $a,n\in \N$ renvoie $a^n$.\\\\
	\begin{minipage}{0.5\linewidth}
		\begin{algorithm}[H]
			\caption{exp($a$, $n$)}
			\eSi{$n=0$}{
				\Retour{$1$}
			}{
				$x =$exp($a$, $n$)\\
				\eSi{$n$ est pair}
				{
					\Retour{$x*x$}
				}
				{
					\Retour{$x*x*a$}
				}
			}
			\end{algorithm}
	\end{minipage} \quad \begin{minipage}{0.4\linewidth}
		Précondition : «\texttt a est un flottant et \texttt n un entier».\\
		Postcondition : $\texttt{exp(a,n)} = a^n$.\\
		Alors, la pré-condition est valide à chaque appel, et comme $a^n = a^{\lceil n \rceil} * a^{\lfloor n \rfloor}$, le principe \ref{1-5} est respecté dans tous les cas donc l'algorithme est partiellement correct.
	\end{minipage}
\end{example}

\section{Correction}

La conjecture de Syracuse dit donc que notre fonction Syracuse termine, et quand elle termine est correcte (i.e. partiellement correcte).

\subsection{Cas général}

\begin{definition}
	Quand un programme termine sur toute entrée valide et est partiellement correct, on dit qu'il est correct, ou encore totalement correct.
\end{definition}

\begin{example}\enspace\\
	\begin{algorithm}[H]
		\caption{fusion($L_1$, $L_2$)}
		$res \gets []$\\
		$i,j \gets 0$\\
		\Tq{
			$i < |L_1|$	et $j < |L_2|$
		}{
			\eSi{$L_1[i] < L_2[j]$}{
				$res$.ajouter($L_1[i]$) \\
				$i \gets i + 1$
			}{
				$res$.ajouter($L_2[j]$) \\
				$j \gets j + 1$
			}
		}
		Ajouter le reste de $L_1$ et de $L_2$ à $res$\\
		\Retour{$res$}
	\end{algorithm}

	\begin{algorithm}[H]
		\caption{tri\_fusion($L$)}
		$n \gets |L|$\\
		\Si{$n\leq 1$}{
			\Retour{$L$}
		}
		$L_1, L_2 \gets$ partionner($L$)\\
		\Retour{fusion( tri\_fusion($L_1$), tri\_fusion($L_2$) )}
	\end{algorithm}
\end{example}

\textbf{Developpement \ref{D?}} Correction totale de tri\_fusion.\\

Néanmoins, ce n'est pas toujours facile. La conjecture de Syracuse est toujours un problème ouvert. Et c'est parfois même pire.

\begin{theorem}
	La correction partielle et la terminaison sont indécidables. \label{1-6}
\end{theorem}

\textbf{Developpement \ref{D?}} Preuve du théorème \ref{1-6}

\subsection{Cas des algorithmes récursifs}

Dans le cas des algorithmes récurifs, on fait régulièrement la correction totale directement.

\begin{proposition}
	SI $(E, \preceq)$ est un ordre bien fondé, alors toutes parties non vides à un élément minimal (plus grand que personne)
\end{proposition}

\begin{theorem}
	\label{1-4}
	Soit $(A, \preceq)$ un ensemble muni d'un ordre bien fondé et $\mathcal P$ une propriété sur $A$, alors $\forall x \in A, \, (\forall y \in A, \, y \preceq A \enspace \Rightarrow \enspace \mathcal P(y) \Rightarrow \mathcal P(x)) \enspace \Rightarrow \enspace \forall x ,\, \mathcal P(x)$
\end{theorem}

\begin{rem}
	Cela étend le principe de réccurence forte sur $\N$
\end{rem}

L'ordre bien fondé nous donne alors le variant et la propriété l'invariant. On montre alors la terminsaison et la correction partielle en même temps.

\begin{example}
	Dans l'exemple \ref{1-6}, la propriété $\mathcal P(n)$ : «exp($a$,$n$)$ = a^n$» vérifie les bypothèses du théorème \ref{1-4}. Donc, $\forall n, \texttt{exp(a,n)} = a^n$, et ce pour tout $a\in \N$
\end{example}

\begin{com}
	Ici on suppose que exp($a$,$n$)$ = a^n$ veut dire que exp termine et renvoie $a^n$
\end{com}

\section{Outils pratiques}

\begin{itemize}[label=$\star$]
	\item Typer : Le fait d'utiliser un typage fort comme en OCamL permet d'éviter beaucoup d'erreurs bêtes
	\item Programmer défensivement en utilisant la bibliothèque assert.h permet de vérifier qu'à un moment donné du code, les hypothèses (ou les invariants) sont satisfaits (et pas seulement une erreur aléatoires parmi 1000 lignes)
	\item Faire des test tout au long de la programmation, en utilisant au maximum la modularité, pour détecter le plus tôt possible les erreurs.
	\item Utiliser des logiciels comme GDB ou valgrind pour détecter les fuites mémoires, ou l'inspecter au cours du programme.
	\item Commentez ! C'est primordial pour déclarer la spécification des fonctions et rendre le code compréhensible, donc déboguable.
\end{itemize}


\begin{com}
	On pourrait développer cette partie également en parlant plus longuement des tests (cf lecon \ref{L3}) qui est un autre type d'outils pour éprouver la correction d'un programme. Mais ca nous emmène un peu loin pour cette leçon (il faut bien faire des choix)
\end{com}
