\dev{Emile Martinez}{}{}{}

\subsection{Principe}
\subsubsection{Definition}

\begin{definition}
	$(\mathcal B, (f_i))$ est une signature sur $X$ si : \begin{itemize}	
	\item $\mathcal B \subset X$ (appelé cas de base)
	
	\item $f_i : X^{\alpha(i)} \to X$ appelé constructeurs, d'arité $\alpha(f_i)$ avec $f_i$ injectif et $\Im(f_i) \cap \Im(f_j) = \empty$ et $\Im(f_i) \cap \mathcal B = \empty$ pour tout $i \neq j$
	\end{itemize}
\end{definition}

\begin{rem}
	On se contente souvent de dire que les constructeurs existent, sans donner leur définition. (de même pour les cas de bases, et pour X) \label{4-3}
\end{rem}

\begin{example}
	\label{4-1}
	On prend une constante $Z$ et un constructeur d'arité 1 $Succ$
\end{example}

\begin{example}
	\label{4-2}
	On peut prendre les constructeurs $\oplus$, $\ominus$ $\otimes$ et $\otimes$ avec $\alpha(\oplus) = 2$, $\alpha(\ominus) = 1$ et $\alpha(\otimes) = 2$ et comme cas de bases $\N$. 
\end{example}

\begin{definition}
	\label{4-4}
	Un ensemble inductif est définit par une signature $(\mathcal B, (f_i))$: \begin{enumerate}
		\item Le plus petit ensemble contenant $\mathcal B$ et stable par tous les $f_i$ (définition par le haut)
		
		ou de manière équivalente
		
		\item $\bigcup T_i$ où $T_0 = \mathcal B$ et $T_{n+1} = T_{n} \cup \bigcup\limits_{i} f_i( T_{n} ^ {\alpha(f_i)})$ (définition par le bas)
	\end{enumerate}
\end{definition}

\begin{example}
	L'ensemble inductif définit par l'exemple \ref{4-1} peut être une définition des entiers naturels \label{4-6}
\end{example}

\begin{example}
	Exemple : L'ensemble inductif $\mathcal A_{simp}$ définit par l'exemple $\ref{4-2}$ est l'ensemble des expressions arithmétiques simplifiées.
\end{example}

\begin{rem}
	$\oplus(1, 1) \neq \otimes(1, 2) \neq \otimes(2, 1)$. On s'intéresse à l'expression et non au résultat.
\end{rem}

\noindent \textbf{Développement \ref{D?}} Validité de la construction d'un ensemble inductif (remarque \ref{4-3} et définition \ref{4-4})

\subsubsection{Induction structurelle}
On prendra maintenant $(\mathcal B, (f_i)_{i \in I})$ une signature.

\begin{proposition}
	\label{4-5}
	Soit $E$ un ensemble inductif construit par $(\mathcal B, (f_i))$\\
	
	Alors la donné de fonction $g_i$ (avec $\alpha(g_i) = \alpha(f_i)$ et $\Im(g_i) \subset Dom(g_j)$ ) et de $f(b)$ pour $b \in \mathcal B$ définit une unique fonction $f$ sur $E$ ayant la propriété:
	$$\forall i \in I, \forall x_1, \, \dots , \, x_{\alpha(i)} \in X, \, f\left(f_i(x_1, \, \dots, \, x_{\alpha(i)})\right) =  g_i\left(f(x_1), \dots, \, f(x_{\alpha(i)})\right)$$
\end{proposition}

\begin{example}
	Sur $\mathcal A_{Simp}$, on peut définir $eval : \mathcal A_{Simp} \to \N$ par \begin{itemize}
		\item $eval(a) = a$ pour $a \in \N$
		\item $eval(\oplus(a, b)) = eval(a) + eval(b)$
		\item $eval(\otimes(a, b)) = eval(a) \times  eval(b)$
		\item $eval(\ominus(a)) = - eval(a)$
	\end{itemize}
\end{example}

\begin{theorem}[Induction structurelle]~
	Soit $E$ l'ensemble inductif définit par $(\mathcal B, (f_i)_{i\in I})$, et $\mathcal P$ une propriété définie pour tout $x\in E$.\\
	
	Alors $\left\{ \begin{array}{cl}
		(i) & \forall b \in \mathcal B, \mathcal P(b)\\
		(ii) & \forall i \in I, \, \forall x_1, \, \dots, x_{\alpha(i)}, \, \left( \forall j, \, \mathcal P(x_j)\right) \implies \mathcal P \big( f_i(x_1, \, \dots, \, x_{\alpha(i)})\big)
	\end{array} \right.$	
\end{theorem}

\begin{rem}
	La récurrence est un cas particulier dans le cas de la définition des entiers par l'exemple \ref{4-6}
\end{rem}

\begin{example}
	On montre par induction structurelle que $eval(e)$ pour $e \in \mathcal A_{simp}$ est multiple du pgcd des constantes apparaissant dans $e$
\end{example}

\begin{definition}
	Un ordre bien fondé est un ordre où toute partie non vide admet un élément minimal (plus grand que personne)
\end{definition}

\begin{proposition}
	L'ordre produit et l'ordre lexicographique d'ordres bien fondés sont bien fondés
\end{proposition}

\begin{example}
	$\N$ avec l'ordre naturel est bien fondé, donc $\N^k$ avec l'ordre produit ou lexicographique aussi.
\end{example}

\begin{theorem}
	Soit $(A, \preceq)$ un ensemble muni d'un ordre bien fondé et $\mathcal P$ une propriété sur $A$, alors $\forall x \in A, \, (\forall y \in A, \, y \preceq A \enspace \Rightarrow \enspace \mathcal P(y) \Rightarrow \mathcal P(x)) \enspace \Rightarrow \enspace \forall x ,\, \mathcal P(x)$
\end{theorem}

\begin{rem}
	Cela étend le principe de réccurence forte.
\end{rem}

\begin{definition}
	Soit $E$ l'ensemble inductif défini par $(\mathcal B, (f_i)_{i \in I})$
	
	On définit l'ordre structurel $\leq_s$ sur $E$ comme la clôture transitive réflexive de $x_j \leq_s f_i(x_1, \, \dots, \, x_{\alpha(i)})$
\end{definition}

\begin{proposition}
	$\leq_s$ est une relation d'ordre bien fondé
\end{proposition}

\begin{corollary}
	On peut alors réécrire l'induction structurelle comme une induction sur l'ordre structurelle
\end{corollary}

\begin{com}
	Si on manque de place, on peut mettre les ordres bien fondés en prérequis (mais alors écrire la formule dans le corollaire)
\end{com}

\subsubsection{En OCaml}

\begin{syntaxe}
	En OcamL on peut créer un type représentant un ensemble inductif avec cette syntaxe :
	\begin{lstlisting}
type  t = Casdebase1 | Casedebase2 | ... 
        | Constructeur1 of type11 *type12 * .... 
        | Constructeur2 of type21 * tpye22 * ...
	\end{lstlisting}
	
	Où : \begin{itemize}
		\item \texttt{Casdebase} peut soit être un type déjà défini d'OcamL, soit une constante (nom commençant par une majuscule)
		\item \texttt{Constructeur} est une étiquette commencée par une majuscule 
		\item \texttt{typei} est est un type OCamL (pouvant contenir \texttt t)
	\end{itemize}
\end{syntaxe}

\begin{example}
	Pour définir les entiers de l'exemple \ref{4-6}, on peut écrire \begin{lstlisting}
type entier = Zero | Succ of entier
	\end{lstlisting}
\end{example}

\begin{syntaxe}
	Pour gérer les types, on peut utiliser le filtrage comme pour les listes.
\end{syntaxe}

\begin{example}
	Pour l'addition sur notre type entier, on peut écrire :
	\begin{lstlisting}
let rec ajouter x y = match y with
  | Zero -> x
  | Succ(z) -> ajouter (Succ(x)) z
	\end{lstlisting}
\end{example}

\begin{example}
	La validité de cette définition vient de la propriété \ref{4-5}
\end{example}

\subsection{Structures de données inductives}

\subsubsection{Les listes chaînées}


En OcamL on peut définir des listes d'entier simplement chaînés par \begin{lstlisting}
type liste = V | Cons of int * liste
\end{lstlisting}

Ainsi, une liste c'est soit une liste vide, soit un entier et le reste de la liste.

\begin{rem}
	Ici \texttt V est le cas de base, et \texttt{Cons} le constructeur. $Cons$ est défini sur $\N \times \{\text{ensemble des listes}\}$ et non $\{\text{ensemble des listes}\}^2$. Cela est un raccourci d'OCamL, où en réalité on définit un constructeur pour chaque premier argument, et donc on construit non pas $Cons(x, l)$ mais $Cons_x(l)$.
\end{rem}

\begin{rem}
	Cela correspond au type \texttt{int list} d'OCamL ;-)
\end{rem}

\begin{exercise}
	Définir inductivement la taille d'une liste chaînée
\end{exercise}

\subsubsection{Les arbres binaires}

\begin{exercise}
	Définir inductivement la taille d'une liste chaînée.
\end{exercise}

\subsubsection{Les arbres binaires}

\begin{definition}[Arbre binaires]
	Soit $A$ un ensemble. On définit de manière inductive les arbres binaires sur $A$ par : \begin{itemize}
		\item l'arbre vide $E$ (cas de base)
		\item si $e\in A$ et $g$ et $d$ sont des arbres binaires, alors $Noeud(e,g,d)$ est un arbre binaire.
	\end{itemize}
\end{definition}

\begin{impl}
	Ce qui en OCamL nous donne \texttt{type 'a arbre = E | Noeud of 'a * 'a arbre * 'a arbre}
\end{impl}

\begin{example}
	La hauteur d'un arbre binaire se calcule alors inductivement par \begin{lstlisting}
let rec hauteur arb = match arb with
  | E -> 0
  | Noeud(e, g, d) -> 1 + max (hauteur g) (hauteur d)
	\end{lstlisting}
\end{example}

\begin{exercise}
	Prouver par induction structurelle la terminaison de la fonction hauteur
\end{exercise}

\begin{exercise}
	Donner la définition inductive de la taille d'un arbre binaire (son nombre d'éléments)
\end{exercise}

\subsubsection{Les arbres généraux}

\begin{definition}
	Un arbre général est un noeud (la racine) et une liste d'arbre (ses fils)
\end{definition}

On voudrait alors définir le type \texttt{arbre} par (pour les arbres d'entier)
\begin{lstlisting}
type arbre = Noeud of int * arbre_liste
\end{lstlisting}
Il faut alors définir le type \texttt{arbre\_liste}, par
\begin{lstlisting}
type arbre_liste = Vide | Cons of arbre * arbre_liste
\end{lstlisting}

On remarque que chaque type a besoin de l'autre pour exister. On écrit alors
\begin{lstlisting}
type arbre = Noeud of int * arbre_liste
and type arbre_liste = Vide | Cons arbre*arbre_liste
\end{lstlisting}

\begin{rem}
	On passe souvent cela sous le tapis grâce au polymorphisme qui définit des manières de construire des types et non des types directement
\end{rem}

\begin{com}
	Dans la défense de plan, on peut parler ici des différentes définition des arbres (par coinduction, avec une infinité de constructeur (pour chaque $k \in \N^*$ d'arité $k$, pour les arbres à $k$ fils), par un constructeur $Ajout\_fils$ d'arité 2 où le premier argument est l'arbre sans son premier fils, et le dernier argument le premier fils (on en déduit les cas de bases))
\end{com}

\subsection{Ensembles inductifs}

\subsubsection{Formules propositionnelles}

\begin{definition}[formule propositionelle]~
	Soit $V$ un ensemble de variables et la signature \begin{itemize}
		\item $\mathcal B = \{ \top, \bot\} \cup V$
		\item le constructeur $\neg$ d'arité 1
		\item les constructeurs $\vee$, $\wedge$ et $\to$ d'arité 2 (en forme infixe)
	\end{itemize}
	Les formules propositionnelles forment l'ensemble inductif défini par cette signature
\end{definition}

\begin{exercise}
	Défini inductivement le nombre de variables présent dans la formule
\end{exercise}

\begin{definition}
	On appelle valuation (ou distribution de vérité) toute fonction $v : V \to \{0, 1\}$
\end{definition}

\begin{definition}[Evaluation d'une formule] Soit $v$ une valuation. La fonction d'évaluation d'une formule $[.]_v : F \to \{0,1\}$ se définit inductivement par \begin{itemize}[label=$\bullet$]
		\item $[\top]_v = 1$
		\item $[\bot]_v = 0$
		\item $[x]_v = v(x)$ pour $x \in V$
		\item $[\neg F]_v = 1 - [F]_v$
		\item $[f_1 \wedge f_2]_v = \min([f_1]_v, [f_2]_v)$
		\item $[f_1 \vee f_2]_v = \max([f_1]_v, [f_2]_v)$
		\item $[f_1 \to f_2]_v = \left\{ \begin{array}{ll}
			0 & \text{si } [f_1]_v = 0 \text{ et } [f_2]_v = 1\\
			1 & \text{sinon}
		\end{array} \right.$
\end{itemize}
\end{definition}

\begin{exercise}
	Définir l'équivalence entre formules et montrer par induction structurelle que, à équivalence près, on peut ne garder que les constructeurs $\neg$ et $\vee$
\end{exercise}

\subsubsection{Langages}

\begin{definition}
	Soit $\Sigma$ un ensemble (appelé alphabet) fini et non vides d'éléments (appelés lettres). On définit inductivement l'ensemble des mots sur $\Sigma$, $\Sigma^*$ par la signature : \begin{itemize}
		\item $\varepsilon$ (le mot vide) comme cas de base
		\item Si $a \in \Sigma$ et $w \in \Sigma^*$, $wa\in \Sigma^*$
	\end{itemize}
\end{definition}

\begin{rem}
	On aurait aussi pu prendre comme définition de $\Sigma^*$, $\bigcup\limits_{n \geq 0} \Sigma^n$
\end{rem}

\begin{definition}
	La concaténation de deux mots $w_1, w_2 \in \Sigma^*$ se définit inductivement comme : \begin{itemize}
		\item $concat(w_1, \varepsilon) = w_1$
		\item $concat(w_1, w_2.a) = concat(w_1, w_2).a$
	\end{itemize}
\end{definition}

\begin{exercise}
	Montrer par induction structurelle que $concat(w_1, concat(a, w_2)) = concat(w_1.a, w_2)$ 
\end{exercise}