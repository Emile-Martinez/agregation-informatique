\dev{Emile Martinez}{}

Quand on aborde un problème à la main, on essaye souvent de construire des solutions au fur et à mesure, en faisant plein de choix locaux. Essayons de mettre cela en oeuvre algorithmiquement.

\section{Algorithmes gloutons}

\subsection{Définition}

\begin{definition}
	Un algorithme glouton construit une solution à un problème par choix successifs considérés localement optimaux, sans jamais revenir en arrière.
\end{definition}

\textbf{Avantage :} C'est souvent peu coûteux, et potentiellement en ligne.

\begin{example}[Gymnases]\enspace
	\begin{itemize}
		\item[Instances :] $n$ évènements, leur date de début $\left\{d_i\right\}_{i \in \{1, \dots, n\}}$ et de fin $\left\{f_i\right\}_{i \in \{1, \dots, n\}}$
		\item[Problème :] trouver un nombre minimal de gymnases pour organiser les évènements
	\end{itemize}
\end{example}

\begin{algo}\enspace
	\begin{enumerate}
		\item Trier les évènements par dates de début croissantes
		\item Allouer successivement le premier gymnase disponible. En ouvrir un si nécessaire.
	\end{enumerate}
\end{algo}