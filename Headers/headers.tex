\documentclass[11pt,a4paper]{book}
\usepackage[utf8]{inputenc}
\usepackage[french]{babel}
\usepackage[left=2cm,right=2cm,top=2cm,bottom=2cm]{geometry}
\usepackage{fancyhdr}
\usepackage{appendix}
\usepackage{amsmath}
\usepackage{amsfonts}
\usepackage{amssymb}
\usepackage{stmaryrd}
\usepackage{mathrsfs}
\usepackage{graphicx}
\usepackage[ruled,vlined, french, algochapter]{algorithm2e}
\usepackage{ntheorem}
\usepackage[table,dvipsnames,svgnames]{xcolor}
\usepackage[framemethod=tikz]{mdframed}
\usetikzlibrary{shadows,shadings}
\usepackage{sectsty}
\usepackage{multicol}
\usepackage{hyperref}
\usepackage{listings}
\usepackage{float}
\usepackage[explicit]{titlesec}
\usepackage{colortbl}
\usepackage{enumitem}
\usepackage{tikz} %Tikz !
\usepackage{multirow}
\usepackage{ulem}
\usepackage{hhline}
\usetikzlibrary{automata, positioning, arrows}
\usetikzlibrary{shapes,shapes.geometric}
\usepackage{cancel}

\usepackage{xcolor}
\definecolor{mGreen}{rgb}{0,0.6,0}
\definecolor{mGray}{rgb}{0.5,0.5,0.5}
\definecolor{mPurple}{rgb}{0.58,0,0.82}
\definecolor{backgroundColour}{rgb}{0.95,0.95,0.92}

\tikzset{elliptic state/.style={draw,ellipse}}

% En-tête et pied de page
\newcommand{\mask}[1]{}

\pagestyle{fancy}
\renewcommand\headrulewidth{1pt}
\fancyhead[L]{Leçons d'Informatique}
\fancyhead[R]{Préparation à l'Agrégation d'Informatique 2022}
\fancyfoot[R]{\tiny $\copyright$ 2024 Martinez 2022 M. Marin}

\setcounter{tocdepth}{1} 



%Design des théorèmes
\usepackage{framed}


\newmdtheoremenv[
outerlinewidth=1pt,
innerlinewidth=0pt,
roundcorner=2pt,
linecolor=black,
shadow=true,
tikzsetting={shading=axis,top color=gray!10},
innertopmargin=1\baselineskip,
skipabove={\dimexpr0.5\baselineskip+\topskip\relax},
needspace=3\baselineskip ,
frametitlefont={\sffamily\bfseries},
]{theorem}%
{\color{BrickRed}Théorème}[chapter]

\newmdtheoremenv[
outerlinewidth=1pt,
innerlinewidth=0pt,
roundcorner=2pt,
linecolor=black,
shadow=true,
tikzsetting={shading=axis,top color=gray!10},
innertopmargin=1\baselineskip,
skipabove={\dimexpr0.5\baselineskip+\topskip\relax},
needspace=3\baselineskip ,
frametitlefont={\sffamily\bfseries},
]{proposition}%
{\color{IndianRed}Proposition}[chapter]

\newmdtheoremenv[
outerlinewidth=1pt,
innerlinewidth=0pt,
roundcorner=2pt,
linecolor=black,
shadow=true,
tikzsetting={shading=axis,top color=gray!10},
innertopmargin=1\baselineskip,
skipabove={\dimexpr0.5\baselineskip+\topskip\relax},
needspace=3\baselineskip ,
frametitlefont={\sffamily\bfseries},
]{lemma}%
{\color{IndianRed}Lemme}[chapter]

\newmdtheoremenv[
outerlinewidth=1pt,
innerlinewidth=0pt,
roundcorner=2pt,
linecolor=black,
shadow=true,
tikzsetting={shading=axis,top color=gray!10},
innertopmargin=1\baselineskip,
skipabove={\dimexpr0.5\baselineskip+\topskip\relax},
needspace=3\baselineskip ,
frametitlefont={\sffamily\bfseries},
]{corollary}%
{\color{IndianRed}Corollaire}[chapter]


\newmdtheoremenv[
outerlinewidth=1pt,
innerlinewidth=0pt,
roundcorner=2pt,
linecolor=black,
shadow=true,
tikzsetting={shading=axis,top color=gray!10},
innertopmargin=1\baselineskip,
skipabove={\dimexpr0.5\baselineskip+\topskip\relax},
needspace=3\baselineskip ,
frametitlefont={\sffamily\bfseries},
]{definition}%
{\color{ProcessBlue}Définition}[chapter]


\newmdtheoremenv[
outerlinewidth=1pt,
innerlinewidth=0pt,
roundcorner=2pt,
linecolor=black,
shadow=true,
tikzsetting={shading=axis,top color=gray!10},
innertopmargin=1\baselineskip,
skipabove={\dimexpr0.5\baselineskip+\topskip\relax},
needspace=3\baselineskip ,
frametitlefont={\sffamily\bfseries},
]{principe}%
{\color{DarkGreen}Principe}[chapter]


\makeatletter
\def\newframedGtheorem#1{%
\theoremprework{
\renewcommand*\FrameCommand{%
  {\color{DimGrey}\vrule width 3pt \hspace{2.5pt}}}
  \framed}%
\theorempostwork{\endframed}%
\newtheorem@i{#1}%
}
\makeatother
%%%


%Définition des différents environnements
\newframedGtheorem{rem}{\color{DimGrey}Remarque}[chapter]
\newframedGtheorem{com}{\color{Blue}Commentaire}[chapter]
\newtheorem{example}{\textbf Exemple}[chapter]
\newtheorem{exercise}{\color{orange}\textbf Exercice}[chapter]
\newtheorem{idee}{\textbf Idée}[chapter]
\newtheorem{temps}{\color{gray} Temps}[chapter]


\newtheorem{notion}{\textbf Notion}[chapter]
\newtheorem{algo}{\textbf Algorithme}[chapter]


\newtheorem{impl}{\textbf Implémentation}[chapter]
\newtheorem{syntaxe}{\textbf Syntaxe}[chapter]
\newtheorem{appl}{\textbf Application}[chapter]
\newtheorem{idea}{\textbf Idée}[chapter]

\newenvironment{proof}[1][\unskip]{\noindent \textit{Démonstration #1. }}{\hfill $\square$ \\}


%Gestion de la hierarchie
\setcounter{secnumdepth}{3}

\renewcommand{\familydefault}{\sfdefault}
\renewcommand{\thesection}{\arabic{section}}
\renewcommand{\thesubsection}{\Roman{subsection}}
\renewcommand{\thesubsubsection}{\Alph{subsubsection}}


\makeatletter
\renewcommand{\@chapapp}{Leçon}
\makeatother


%En-tête des leçons 
% utilisation : \debut{Auteurs}{niveau}{pré-requis}{références}
\newcommand\debut[4]{
\noindent \textbf{Auteur\textperiodcentered e\textperiodcentered s:} #1  \\
\noindent\textbf{Niveau :} #2 \\
\noindent\textbf{Pré-requis :} #3 \\
\noindent\textbf{Références :} #4 \\
}

%En-tête des développements 
% utilisation : \debut{Auteurs}{références}
\newcommand\dev[2]{
\noindent \textbf{Auteur\textperiodcentered e\textperiodcentered s:} #1  \\
\noindent\textbf{Références :} #2 \\
}

%Commande personnalisées
\newcommand{\N}{\ensuremath{\mathbb{N}}}
\newcommand{\overarrow}[1]{\smash{\overset{#1}{\longrightarrow}}}
\newcommand*\circled[1]{\tikz[baseline=(char.base)]{
		\node[shape=circle,draw,inner sep=2pt] (char) {#1};}}

%Pour le listing de code
\lstset{
frame = single, 
framexleftmargin=1pt,
literate=
{à}{{\`a}}1
{é}{{\'e}}1
{è}{{\`e}}1
{ô}{{\^o}}1}

\lstset{escapeinside={<@}{@>}}

\definecolor{lightpink}{HTML}{f7d1d5}
%Éviter le saut de page après un chapitre
\renewcommand{\cleardoublepage}{\newpage}
\newcommand\countme{\refstepcounter{\thechapter}\thechapter}

\lstdefinestyle{CStyle}{
	backgroundcolor=\color{backgroundColour},   
	commentstyle=\color{mGreen},
	keywordstyle=\color{magenta},
	numberstyle=\tiny\color{mGray},
	stringstyle=\color{mPurple},
	basicstyle=\footnotesize,
	breakatwhitespace=false,         
	breaklines=true,                 
	captionpos=b,                    
	keepspaces=true,                 
	numbers=left,                    
	numbersep=5pt,                  
	showspaces=false,                
	showstringspaces=false,
	showtabs=false,                  
	tabsize=2,
	language=C
}
%Copyright


\newlength\mylen
\newcommand\multientree[1]{%
	\settowidth\mylen{\Entree{}}%
	\setlength\hangindent{\mylen}%
	\hspace*{\mylen}#1\\}