
\dev{Marin Malory}{\cite{Sipser}}

\textit{
Ce développement propose d'étudier des résultats de décidabilité et d’indicibilité autour des langages rationnels, permettant d'illustrer de manière intéressante la hiérarchie de Chomsky. Il s'insère naturellement dans la leçon \ref{L27} ainsi que dans la leçon \ref{L29} si les concepts de décidabilité sont abordés.}

\paragraph{Introduction.} On considère trois problèmes autours des langages rationnels. Deux d'entre eux sont décidables, et l'autre est indécidable.

\paragraph{Automate et langage vide.}
$$
E_{AFD} = \{\langle A \rangle | A \text{ est un AFD et } L(A)=\emptyset \} 
$$

\begin{theorem}
$E_{AFD}$ est décidable.
\end{theorem}

\begin{proof}
Un automate fini déterministe accepte au moins un mot si, et seulement si, il peut atteindre un état final depuis l'état initial en suivant la fonction de transition.\newline

\fbox{\begin{minipage}{0.9\linewidth}
\noindent $\mathbf{T=}$ \og Sur l'entrée $\langle A\rangle$ :
\begin{enumerate}
\item Marquer l'état initial de $A$.
\item Répéter tant qu'un nouvel état est marqué :
\item ~~ Marquer chaque état qui a une transition depuis n'importe quel état marqué.
\item Si aucun état final n'est marqué, \textbf{accepter}, sinon \textbf{rejeter}. \fg
\end{enumerate}
\end{minipage}}

\end{proof}


\paragraph{Égalité de langages rationnels.} On considère le problème suivant :
$$
EQ_{AFD} =\{\langle A,B \rangle | \text{$A$ et $B$ sont des AFDs et } L(A)=L(B) \} 
$$

\begin{theorem}
$EQ_{AFD}$ est décidable.
\end{theorem}

\begin{proof}
On va montrer ce théorème à l'aide du précédent. On construit un nouvel automate $C$ à partir de $A$ et $B$ qui accepte seulement les mots qui sont soit dans $L(A)$, soit dans $L(B)$, mais pas les deux.

Le langage de $C$ est défini par :
$$
L(C)=(L(A) \cap \overline{L(B)}) \cup ( \overline{L(A)} \cap L(B))
$$
\begin{rem}
On dit que $L(C)$ est la \textbf{différence symétrique} de $L(A)$ et $L(B)$.
\end{rem}

Montrons que $L(A)=L(B)$ si, et seulement si, $L(C)=\emptyset$.
Si $L(A)=L(B)=X$, alors $L(C)=(X\cap \overline{X})\cup ( \overline{X} \cap X) = \emptyset \cup \emptyset = \emptyset$. Réciproquement, si $L(C)=\emptyset$, on a $L(A) \cap \overline{L(B)}=\emptyset$ et  $( \overline{L(A)} \cap L(B))=\emptyset$. Soit $x\in L(A)$, alors par l'absurde, si $x\notin L(B)$, alors $x\in \overline{L(B)}$ et donc $x\in L(A) \cap \overline{L(B)}$, ce qui est absurde. Ainsi, on a $L(A)\subset L(B)$ et de la même manière $L(B)\subset L(A)$ ce qui conclut la preuve. 

On peut alors construire $C$ via $A$ et $B$ en utilisant les résultats de clôtures sur les langages rationnels. En effet, ces constructions sont des algorithmes pouvant être implantées via des machines de Turing. On en déduit la machine $F$ qui décide $EQ_{AFD}$ :\newline

\fbox{\begin{minipage}{0.9\linewidth}
\noindent $\mathbf{F=}$ \og Sur l'entrée $\langle A,B\rangle$ :
\begin{enumerate}
\item Construire $C$ comme décrit ci-dessus.
\item Simuler $T$ sur l'entrée $\langle C \rangle$.
\item Si $T$ accepte, \textbf{accepter}. Si $T$ rejette, \textbf{rejeter}. \fg
\end{enumerate}
\end{minipage}}

\end{proof}


\paragraph{Un problème indécidable.} On considère le problème suivant :
$$
\mathbf{R_{TM}}=\{\langle M \rangle | \text{$M$ est une MT et } L(M) \text{ est rationnel}\}
$$

\begin{theorem}
$\mathbf{R_{TM}}$ est indécidable.
\end{theorem}

\begin{proof}
On procède par réduction depuis le problème de l'acceptation $\mathbf{A_{TM}}$. On suppose que $\mathbf{R_{TM}}$ est décidable par une machine de Turing $\mathbf{R}$ et on l'utilise pour construire une machine de Turing $\mathbf{S}$ qui décide $\mathbf{A_{TM}}$.\newline

\noindent\textbf{Idée.} $\mathbf{S}$ reçoit en entrée $\langle M,w\rangle$ où $M$ est une machine de Turing et $w$ un mot, et va modifier $M$ en une nouvelle machine $M_2$ dont le langage est rationnel si, et seulement si $M$ accepte $w$.
Pour cela, $M_2$ va reconnaître automatique tous les mots de $\{0^n1^n | n\geq 0\}$, mais si en plus $M$ accepte $w$, alors $M_2$ accepte tous les autres mots.\newline

\noindent \textbf{Construction de $\mathbf{S}$.} \newline

\fbox{\begin{minipage}{0.9\linewidth}
\noindent $\mathbf{S=}$ \og Sur l'entrée $\langle M,w\rangle$ :
\begin{enumerate}
\item Construire  MT $M_2$ suivante :

~~\fbox{\begin{minipage}{0.8\linewidth}
$\mathbf{M_2=}$ \og Sur l'entrée $x$ :
\begin{enumerate}
\item Si $x$ est de la forme $0^n1^n$, \textbf{accepter}.
\item Sinon, simuler $M$ sur l'entrée $w$ et \textbf{accepter} si $M$ accepte $w$.\fg
\end{enumerate}
\end{minipage}}
\item Simuler $\mathbf{R}$ sur l'entrée $\langle M_2 \rangle$.
\item Si $\mathbf{R}$ accepte, \textbf{accepter} ; si $\mathbf{R}$ rejette, \textbf{rejeter}.\fg
\end{enumerate}
\end{minipage}}\newline

Montrons que $\mathbf{S}$ décide $\mathbf{A_{TM}}$. Soit $M$ une machine de Turing et $w$ un mot.

Si $M$ accepte $w$, alors on remarque que $L(M_2)= \Sigma^*$, car $M_2$ accepte sur toute entrée $x$. Ainsi, $L(M_2)$ est rationnel, $\mathbf{R}$ accepte sur l'entrée $\langle M_2 \rangle$ et finalement $\mathbf{S}$ accepte.

Si $M$ n'accepte pas $w$, on remarque que $L(M_2)= \{0^n1^n | n\geq 0\}$ qui n'est pas rationnel. Ainsi $\mathbf{R}$ rejette sur l'entrée $\langle M_2 \rangle$ et donc $\mathbf{S}$ rejette.

Finalement, $\mathbf{S}$ décide $\mathbf{A_{TM}}$, et ainsi $\mathbf{R_{TM}}$ est indécidable.
\end{proof}

\begin{rem}
Ce résultat nous montre qu'il n'est pas possible de décider si, lors de notre choix d'une machine de Turing comme modèle de calcul, on aurait pu simplement choisir un automate.
\end{rem}

\begin{rem}
Ce résultat est en fait un corollaire du théorème de Rice.
\end{rem}

\begin{rem}
Le fait que $\{0^n1^n[n\geq 0\}$ n'est pas rationnel découle du lemme de l'étoile, dont on rappelle l'énoncé ci-dessous.
\end{rem}

\begin{lemma}[de l'étoile]
Soit $L$ un langage rationnel. Il existe un entier $p$ tel que pour tout mot $w$ de $L$ tel que $|w|\geq p$ possède une factorisation $w=xyz$ vérifiant :
\begin{enumerate}
\item pour tout $i\geq 0, xy^iz\in L$ ;
\item $|y|>0$ ;
\item $|xy|\leq p$.
\end{enumerate}
\end{lemma}
