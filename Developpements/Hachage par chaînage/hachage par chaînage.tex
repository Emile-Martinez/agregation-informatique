\dev{Rousseau Guillaume, Marin Malory}{\cite{Cormen}}

\textit{Ce développement étudie la complexité moyenne de la recherche dans une table de hachage où les collisions sont gérées par chaînage. Il trouve ainsi sa place dans la leçon \ref{L6}, et peut servir d'illustration dans la leçon \ref{L4}. Ce développement étant un peu cours, il pourra être complété par une simple analyse du pire des cas au début, ainsi qu'avec des dessins au tableau afin d'illustrer les démonstrations.}

\paragraph{Introduction.} On considère une table de hachage où les collisions sont résolues par chaînage, et on se place dans l'hypothèse de \textbf{hachage uniforme simple} : chaque élément a la même chance d'être haché vers l'une des alvéoles indépendamment des endroits où les autres éléments sont allés.

Pour une table de hachage à $m$ alvéoles et $n$ éléments, on note $\alpha = n/m$ le facteur de remplissage.

\begin{theorem}
Dans une table de hachage pour laquelle les collisions sont résolues par chaînage, une recherche infructueuse prend un temps moyen $\Theta(1+\alpha)$, sous l'hypothèse d'un hachage simple uniforme.
\end{theorem}

\begin{proof}
On note $\mathbf{Rech}^-(T,k)$ la variable aléatoire qui compte le nombre de comparaisons dans une recherche infructueuse d'un élément $k$ dans un tableau $T$ à $n$ éléments.
$$
\mathbb{E}(\mathbf{Rech}^-(T,k)) = \sum_{i=1}^m \mathbb{P}(h(k)=i) \lambda_i
$$
où $\lambda_i$ est la taille de l'alvéole $i$ de $T$.

Par hypothèse de hachage uniforme simple, 
$$
\mathbb{E}(\mathbf{Rech}^-(T,k)) = \frac{1}{m} \sum_{i=1}^m \lambda_i = \alpha
$$

Maintenant, on note $\mathbf{Rech}^-(n,m)$ le nombre de comparaison dans un tableau déjà rempli avec $n$ éléments $k_1,...,k_n$. Par hypothèse de hachage uniforme simple, le tableau est tiré uniformément parmi l'ensemble des tableaux à $m$ alvéoles et $n$ éléments. Ainsi, par linéarité de l'espérance,
$$
\mathbb{E}(\mathbf{Rech}^-(n,m)) =\frac{1}{m^n}\sum_{T} \mathbb{E}(\mathbf{Rech}^-(T,k))
$$ 
et donc $\mathbb{E}(\mathbf{Rech}^-(n,m))=\alpha$.
\end{proof}

\begin{theorem}
Dans une table de hachage pour laquelle les collisions sont résolues par chaînage, une recherche fructueuse prend un temps moyen $\Theta(1+\alpha)$, sous l'hypothèse d'un hachage simple uniforme.
\end{theorem}

\begin{proof}
L'idée est de prendre aléatoirement un élément qui est déjà dans le tableau, et compter le nombre de comparaisons nécessaires. Le nombre d'éléments examinés pour un élément $x$ est $1$ plus le nombre d'éléments suivants qui ont été hachés au même endroit.

On suppose que l'on a inséré les éléments $x_1,...,x_n$ dans la table de hachage, de clés respectives $k_1,...,k_n$. On note $\mathbf{Rech}^+(x_i)$ le nombre de comparaisons nécessaires pour trouver $x_i$. De plus, on note $X_{ij} = \mathbf{1}\{h(k_i)=h(k_j)\}$. Par hypothèse de hachage simple uniforme, $\mathbb{E}(X_{ij})=1/m$.
Ainsi, 

\begin{align*}
\mathbb{E}(\mathbf{Rech}^+(n,m)) &= \frac{1}{n} \sum_{i=1}^n \mathbb{E}(\mathbf{Rech}^+(x_i)) \\
&= \frac{1}{n} \sum_{i=1}^n \mathbb{E}\left( 1+\sum_{j=i+1}^n X_{ij}\right) \\
&= \frac{1}{n} \sum_{i=1}^n \left( 1+\sum_{j=i+1}^n \mathbb{E}(X_{ij})\right)\\
&= 1+ \frac{1}{n}\sum_{1\leq i <j \leq n}\frac{1}{m} \\
&= 1 + \frac{1}{n}\frac{n(n-1)}{2m} \\
&= 1 + \frac{\alpha}{2}- \frac{1}{2m}
\end{align*}
\end{proof}
