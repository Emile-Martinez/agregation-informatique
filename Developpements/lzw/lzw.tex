\dev{Emile Martinez}{Balabonski MP2I (notamment les quantifications à la fin)}

\paragraph{Idée} Compresser les répétitions de sous-chaines.\\
$\to$ Cette algorithme est online

\begin{com}
	Mentionner la différence avec Huffman qui est pas online et qui fonctionne pas sur les sous-chaines mais directement sur les caractères. Et que si on faisait huffman sur les mots, on aurait un très gros truc a rajouter au début, et en plus le caractère ' ' serait arbitraire.
\end{com}

\begin{enumerate}
	\item On associe un entier à chaque lettre de notre alphabet \quad $\to$ dictionnaire
	\item On parcourt le mot pas encore dans le dictionnaire : \begin{itemize}
		\item On met dans le dictionnaire le plus grand préfixe possible
		\item On ajoute au dictionnaire une nouvelle clé : préfixe + lettre suivante (nouveau numéro)
	\end{itemize}
\end{enumerate}

%\paragraph{Exemple} "entendent", $\Sigma = \{\underset{0}{E}, \underset{1}{N}, \underset{2}{T}, \underset{3}{D}\}$
%
%\begin{tabular}{r|c|c}
%	 entrée \quad \quad & résultat & dictionnaire \\
%	ENTENDENT & 0 & EN $\to$ 4 \\
%	NTENDENT & 1 & NT $\to$ 5 \\
%	TENDENT & 2 & TE $\to$ 6 \\
%	ENDENT & 4 & END $\to$ 7 \\
%	DENT & 3 & DE $\to$ 8 \\
%	ENT & 4 & ENT $\to$ 9 \\
%	T & 2 &
%\end{tabular}\\
%
%Décompression :\\
%\begin{tabular}{c|c|c}
%	entrée & résutlat & dictionnaire \\
%	0 & E & EN $\to$ 4 \\
%	1 & N & NT $\to$ 5 \\
%	2 & T & TE $\to$ 6 \\
%	4 & & \\
%	3 & & \\
%	4 & & \\
%	2 & & \\
%\end{tabular}\\
%
%\paragraph{Problème :} $\Sigma = \{\underset{0}{L}, \underset{1}{A}, \underset{2}{E}, \underset{3}{R}\}$ LALALALALERE$\to$0,1,4,6,5,2,3,2
%
%\begin{tabular}{c|c|c}
%	0 & L & \\
%	1 & A & LA $\to$ 4 \\
%	4 & LA & AL $\to$ 5 \\
%	6 & LAL & LAL $\to$ 6 \\
%\end{tabular}
%


\paragraph{Exemple} "ALLALALA", $\Sigma = \{\underset{0}{\texttt A}, \underset{1}{\texttt B}\}$\\
\begin{tabular}{r|c|l}
	\texttt{ALLALALA} & Sortie & Dictionnaire \\
	&& \texttt A $\to$ 0\\
	&& \texttt L $\to$ 1\\
	\texttt{\emph{A}L\enspace\enspace\enspace\enspace\enspace\enspace} &0& \texttt{AL} $\to$ 2\\
	\texttt{\emph{L}L\enspace\enspace\enspace\enspace\enspace} &1& \texttt{LL} $\to$ 3\\
	\texttt{\emph{L}A\enspace\enspace\enspace\enspace} &1& \texttt{LA} $\to$ 4\\
	\texttt{\emph{AL}A\enspace\enspace} &2& \texttt{ALA} $\to$ 5\\
	\texttt{\emph{ALA}} &5&\\
\end{tabular}


\begin{com}
	Essayons de faire un algorithme de décompression
\end{com}

\begin{tabular}{c|c|l}
	Entrée & Sortie & Dictionnaire \\
	& & \texttt 0 $\to$ A \qquad \texttt 1 $\to$ L\\
	0 & \texttt A & \\
	1 & \texttt L & 2 $\to$ \texttt{AL}\\
	1 & \texttt L & 3 $\to$ \texttt{LL}\\
	2 & \texttt L & 4 $\to$ \texttt{LA}\\
\end{tabular}


\begin{com}
	Ca a l'air de fonctionner. Essayons d'écrire un code pour cela.\\
	(Quand on le fait, laisser de la place au milieu pour pouvoir faire les modifications a venir dessus)
\end{com}

\begin{algorithm}[H]
	$d \gets$ dictionnaire initial\\
	$vieux\_i \gets None$\\
	$indice \gets |d|$\\
	\Tq{il y a une entrée}{
		$i \gets input()$\\
		$print(d[i])$\\
		\Si{$vieux\_i \neq None$}{
			$d[indice] \gets d[vieux\_i] + d[i][0]$\\
		}
		$vieux\_i \gets i$\\
		$indice ++$\\
	}
\end{algorithm}

\begin{com}
	Reprenons l'execution sur l'exemple
\end{com}

\paragraph{Problème} On a pas 5 dans le dictionnaire. On a réutiliser tout de suite le facteur que l'on a ajouter. \textcolor{red}{Il faut traiter à part ce cas là.}

\begin{com}
	Bien expliquer sur la compression, que c'est parce que on a rajouter 5 puis utiliser 5. Que on a le problème parce que on ajoute dans le dico avec un de décalage à la decompression. Que par conséquent le problème vient du fait que on reconnu le motif que l'on vient d'ajouter, donc on connait la lettre suivante, c'est celle du mot d'avant.
\end{com}

\begin{algorithm}[H]
	$d \gets$ dictionnaire initial\\
	$vieux\_i \gets None$\\
	$indice \gets |d|$\\
	\Tq{il y a une entrée}{
		$i \gets input()$\\
		\eSi{\color{red} $i \in d$}{
			$print(d[i])$\\
			\Si{$vieux\_i \neq None$}{
				$d[indice] \gets d[vieux\_i] + d[i][0]$\\
			}
		}{
			\color{red}
			$d[indice] \gets d[vieux\_i] + d[vieux\_i][0]$\\
			$print(d[indice])$\\
		}
		$vieux\_i \gets i$\\
		$indice ++$\\
	}
\end{algorithm}




\paragraph{Implementation}\begin{itemize}
	\item Représentation des entiers \begin{itemize}[label=$\to$]
		\item Nombre limité de bits (ex. 12) mais dico limité.\\
		$\to$ On arrête de le compléter \\
		$\to$ On vide et on recommence
		\item On augmente la taille des représentations au cours de l'algo\\
	\end{itemize}
	\item Choix de l'alphabet \begin{itemize}
		\item $\{0,1\}$
		\item les 256 char possibles
	\end{itemize}
\end{itemize}


\begin{com}
	Sur les tour du monde en 80 jours, avec l'alphabet $\{0, 1\}$, 29\%, avec l'alphabet de tous les char 57\%, en fixant la taille du dictionnaire à 16 (moins, on détecte moins de répétition, plus on a trop de bits pour trop peu de facteurs). Avec Huffman, on faisait 44\%.
\end{com}

\paragraph{Question} Gagne-t-on toujours ?\\
Compression : $LZW \Sigma^* \to \Sigma^*$. A-t-on alors, $\forall w \in \Sigma^*, \, |LZW(w)| \leq |w|$.\\
$fLZW$ est injectif donc non (en effet, $\left|LZW\left(\Sigma^n\right)\right| = \left|\Sigma^n\right|> \left| \Sigma^{n-1}\right| $).\\

Donc LZW, des fois, on y perd. \texttt{0011} on aura perdu, car on devra stocker nos facteurs sur 2 bits, mais en ne codant que des lettres (7 bits en tout au lieu de 4).

\paragraph{Remarque} On peut choisir de ne perdre au plus que 1 bit.\raisebox{-0.5\height}{
$\begin{array}{rccl}
	f:&\Sigma^* & \to & \Sigma^*\\
	& w & \mapsto & \left\{ \begin{array}{ll}
		\texttt 0.\text{LZW}(w) & \text{si } |\text{LZW}(w)| < |w|\\
		\texttt 1.w & \text{sinon}
	\end{array} \right.
\end{array}$}