\dev{Emile Martinez, Malory Marin}{}

\begin{definition}[2-$\mathbf{SAT}$]~
	
	\noindent \textbf{Données :} 
	\begin{itemize}
		\item un ensemble de variables propositionnelles $X = \{x_1,...,x_n\}$
		\item une formule $\varphi$ sous forme normale conjonctive où chaque clause est composée de $2$ littéraux (un littéral étant une variable ou sa négation).
	\end{itemize}
	\noindent\textbf{Problème :} existe-t-il une valuation $ v : \{x_1,...,x_n\} \rightarrow \{0,1\}$ telle que $v(\varphi)=1$ ?
\end{definition}

\begin{rem}
	$y_1 \vee y_2 \quad \equiv \quad \overline{y_1} \to y_2 \quad \equiv \quad \overline{y_2} \to y_1$ en identifiant $\overline{\overline{x}}$ à $x$
\end{rem}

Associons à $\varphi$ le graphe $G = (S,A)$ $$\begin{array}{l}
	S = \left\{ x_1, \, \dots x_n, \, \overline{x_1}, \, \dots, \, \overline{x_n} \right\} \\
	A = \left\{ (y_1, y_n) \enspace \big/\enspace  y_1, y_2 \in S, \enspace \varphi \text{ contient une clause équivalente à } y_1 \to y_2 \right\}
\end{array}$$

\begin{rem}
	On remarque que le graphe est de taille linéaire en la taille de la formule.
\end{rem}

\begin{example}
	$\varphi = \enspace (x \vee \overline y) \enspace \wedge \enspace (\overline x \vee y) \enspace \wedge \enspace (\overline x \vee \overline y) \enspace \wedge \enspace (x \vee \overline z)$\\
	
	\begin{tikzpicture}[->, node distance = 2cm]
		\node[state] (q0) {$z$};
		\node[state, right of = q0] (q1) {$x$};
		\node[state, right of = q1] (q2) {$\overline y$};
		\node[state, below of = q1] (q3) {$y$};
		\node[state, below of = q2] (q4) {$\overline x$};
		\node[state, right of = q4] (q5) {$\overline z$};

		\draw (q0) edge[] node{} (q1);
		\draw (q1) edge[] node{} (q2);
		\draw (q3) edge[] node{} (q4);
		\draw (q1) edge[bend left] node{} (q3);
		\draw (q3) edge[bend left] node{} (q1);
		\draw (q2) edge[bend left] node{} (q4);
		\draw (q4) edge[bend left] node{} (q2);
		\draw (q4) edge[] node{} (q5);

	\end{tikzpicture}
\end{example}

\begin{proposition}
	$\varphi$ est satisfiable $\Leftrightarrow$ pour tout $x\in \{x_1, \, \dots, \, x_n\}$, $x$ et $\overline x$ ne sont pas dans la même composante connexe.
\end{proposition}

\begin{proof} \enspace \\
	
	$\boxed{\Rightarrow}$ Exercice
	 \begin{com}
	 	On ne le fait pas par manque de temps, mais on pourrait virer l'exemple par exemple et faire quand même ce sens.
	 \end{com}
	
	$\boxed{\Leftarrow}$ Supposons que pour tout $x\in \{x_1, \dots, x_n\}$, $x$ et $\overline x$ ne sont pas dans la même composante connexe (i.e. la droite de $\Leftrightarrow$)
	\paragraph{Idée} On va construire une valuation pour $\varphi$.\\
	
	Considérons le graphe $G_c$ des composantes fortement connexes de $G$ qui orienté acyclique (DAG)
	
	\begin{example}
		\raisebox{-0.5\height}{\begin{tikzpicture}[->, node distance = 2.5cm]
			\node[state] (q0) {$z$};
			\node[state, right of = q0] (q1) {$x, y$};
			\node[state, right of = q1](q2) {$\overline x, \overline y$};
			\node[state, right of =q2](q3) {$\overline z$};
			
			\draw (q0) edge[] node{} (q1);
			\draw (q1) edge[] node{} (q2);
			\draw (q2) edge[] node{} (q3);
			
		\end{tikzpicture}}
	\end{example}

	\begin{lemma}
		Si $C$ est un noeud de $G_C$, il existe $\overline C$ un noeud de $G_C$ tel que $\overline C = \{\overline x / x \in C\}$
	\end{lemma}

	\begin{proof}
		Soit $u,v \in C$. Alors $\overline u, \overline v$ dans la même composante connexe (car $u\to y \to \dots \to v \implies \overline v \to \dots \to \overline y \to \overline u0$). Donc $\exists \overline C \in G_c : \{\overline u / u \in C\} \subset \overline C$ d'où le résultat par symétrie.
	\end{proof}
	
	On procède alors par induction sur le nombre de sommets de $G$ (sur le fait qu'il existe une valuation respectant $G$, donc $\varphi$)
	
	\begin{itemize}[label=$\star$]
		\item Pour $0$ une telle valuation (c'est un fonction de $\emptyset \to \{0,1\}$)
		\item $G_c$ est un DAG donc $\exists P \in V_c : d_+(P) = 0$. $P$ est donc un puit et $\overline P$ une source. Par induction, on prend alors une valuation $v$ pour $G\setminus P \cup \overline P$.
		
		On étend alors $v$ par pour $y \in P, v(y) = 1$\\
		\quad $\to$ ceci est légale car pour $y \in P$ et $z\in P$, on a pas $\overline y = z$ (par hypothèse)\\
		
		On a alors pour $y \in \overline P, v(y) = 0$\\
		
		On a alors 5 types d'arêtes dans $G$ : \begin{enumerate}
			\item les arêtes de $\overline P$ dans $\overline P$ qui sont respectées par $v$ (car $0->0$)
			\item les arêtes de $\overline P$ dans $S\setminus P \cup \overline P$ qui sont respectées (car faux implique tout)
			\item les arêtes de $S\setminus P \cup \overline P$ dans $S\setminus P \cup \overline P$ qui sont respectées (par hypothèse d'induction)
			\item les arêtes de $S\setminus P \cup \overline P$ dans $P$ qui sont respectées car tout implique vrai.
			\item les arêtes de $P$ dans $P$ qui sont respectées (car $1 \to 1$)
		\end{enumerate}
		Donc $v$ est une valuation respectant $G$.
	\end{itemize}
	
\end{proof}

Ainsi, pour résoudre 2-SAT il suffit de calculer les composantes fortement connexes du graphe associé à la formule grâce à l'algorithme de Kosaraju. On fait ensuite un tableau où on met l'identifiant de la composante connexe de chaque noeud et de son complémentaire, et on vérifie qu'ils n'ont pas la même.