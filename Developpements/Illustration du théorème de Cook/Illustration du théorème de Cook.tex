\dev{Emile Martinez}{}

\textit{Illustration du théorème de Cook et de la puissance du problème SAT en présentant sur des exemples (3-coloration et Subset-Sum) comment se ramener au problème SAT. Il peut illustrer les leçons de NP-complétude ou les leçons parlant de formules propositionnelles.}

\begin{definition}
	Le problème \textbf{coloration} est un problème sur les graphes non orientés.
	
	Une instance positive de coloration est un graphe $G = (S,A)$ et un entier $k$ tel que
	$$\exists c:S \rightarrow  \llbracket1, k \rrbracket \enspace : \enspace \forall (u,v)\in A, \enspace c(u) \neq c(v)$$

	On dit alors que $c$ est une $k$-coloration de $G$.

\end{definition}

\begin{example} Sur le graphe suivant :
	\begin{center}
		\begin{tikzpicture}[-, node distance=2.5cm]
			\node[state] (q0) {};
			\node[state, below left of = q0] (q1) {};
			\node[state, below right of = q0] (q2) {};
			\node[state, below right of = q1] (q3) {};
			\node [state, above right of = q2] (q4) {};
			
			
			\draw (q0) edge[] node{} (q1) ;
			\draw (q1) edge[] node{} (q2) ;
			\draw (q1) edge[] node{} (q3) ;
			\draw (q2) edge[] node{} (q4) ;
			\draw (q2) edge[] node{} (q0) ;
			\draw (q2) edge[] node{} (q3) ;
			
		\end{tikzpicture}
	\end{center}

	\begin{tabular}{c|c}
				\begin{tikzpicture}[-, node distance=2.5cm]
		\node[state] (q0) {1};
		\node[state, below left of = q0] (q1) {2};
		\node[state, below right of = q0] (q2) {1};
		\node[state, below right of = q1] (q3) {3};
		\node [state, above right of = q2] (q4) {2};
		
		
		\draw (q0) edge[] node{} (q1) ;
		\draw (q1) edge[] node{} (q2) ;
		\draw (q1) edge[] node{} (q3) ;
		\draw (q2) edge[] node{} (q4) ;
		\draw (q2) edge[] node{} (q0) ;
		\draw (q2) edge[] node{} (q3) ;
		
		\end{tikzpicture} & \begin{tikzpicture}[-, node distance=2.5cm]
		\node[state] (q0) {1};
		\node[state, below left of = q0] (q1) {2};
		\node[state, below right of = q0] (q2) {3};
		\node[state, below right of = q1] (q3) {1};
		\node [state, above right of = q2] (q4) {2};
		
		
		\draw (q0) edge[] node{} (q1) ;
		\draw (q1) edge[] node{} (q2) ;
		\draw (q1) edge[] node{} (q3) ;
		\draw (q2) edge[] node{} (q4) ;
		\draw (q2) edge[] node{} (q0) ;
		\draw (q2) edge[] node{} (q3) ;
		
		\end{tikzpicture} \\
		n'est pas une 3-coloration  & est une 3-coloration\\
	\end{tabular}
	
\end{example}

\begin{theorem}
	Il existe une transformation polynomiale d'une instance de coloration vers une instance de SAT.
\end{theorem}

\begin{proof}
	Soit $G = (S,A)$ un graphe et $k \in \mathbb N^*$ une instance de coloration.
	
	Créons les variables $x_{v,i}$ pour $v \in S$ et $i \in \llbracket1, k \rrbracket$, dont on voudra donner la signification $x_{v,i}$ dira si $v$ est à la couleur $i$.
	
	\paragraph{Existence d'une couleur}
	Créons pour $v \in S$, $E_v = \bigvee\limits_{i = 1}^k x_{v,i}$\newline
	Ainsi, si $E_v$ est satisfaite, au moins un des $x_{v,i}$ est à vrai.
	\paragraph{Unicité d'une couleur}
	Créons pour $v \in S$, $U_v = \bigwedge_{i = 1}^k \bigwedge_{j = 1 \\ j \neg i}^k \neg x_{v, i} \vee \neg x_{v,j} $\newline
	Ainsi, si $U_v$ est satisfaite, on ne peut pas avoir deux $x_{v, i}$ différents qui sont satisfaits. On en a donc au plus un.
	\paragraph{Coloration}
	Créons $C = \bigwedge\limits_{(u,v)\in A} \bigwedge\limits_{i = 1}^k \neg x_{u,i} \vee \neg x_{v,i}$\newline
	Si $C$ est satisfaite, pour aucune couleur, deux voisins dans le graphe on leur variable à vrai. Dans notre interprétation, deux voisins ne peuvent pas avoir la même couleur.
	
	\paragraph{} On créer alors l'instance de SAT $I_{SAT} = \bigwedge\limits_{v\in S} E_v \wedge \bigwedge\limits_{v \in S} U_v \wedge C$
	
	\paragraph{ $\boxed{\implies}$ } Supposons que $G$ admette une $k$-coloration $c$. Alors on choisit comme valuation $\sigma (x_{v,i}) = V \iff c(v) = i$. Alors, $c$ étant une fonction, $\sigma$ évalue bien à vrai $E_v$ et $U_v$ (car il y a un et un seul $i$ tel que $c(v) = i$), et étant une coloration, $\sigma$ évalue à vrai $C$ car si $(u,v)\in A, \neg \left( c(u) = i \wedge c(v) = i \right)$ (car $c(u) \neq c(v)$)
	
	\paragraph{$\boxed{\impliedby } $} Supposons que $\sigma$ évalue $I_{SAT}$ à vrai.
	Alors, pour $v \in S$, $U_v$ et $E_v$ nous disent qu'un unique $x_{v,i}$ est à vrai. 
	$$\forall v \in S, \, \exists! i \in  \llbracket1, k \rrbracket \enspace : \enspace \sigma(x_{v,i}) = V$$
	Notons $c(v)$ cet unique $i$. Alors par existence et unicité de $i$, $c : S \rightarrow \llbracket1, k \rrbracket $ est bien défini. De plus, si $(u,v) \in A$, alors $\sigma$ satisfaisant $C$, $\sigma\left(x_{u,c(u)}\right) = V \implies \sigma \left(x_{v, c(u)}\right) = F$ donc $c(v) \neq c(u)$.
	
	Ainsi, $I_{SAT}$ est une instance positive de $SAT$ si et seulement si $G, k$ en est une de coloration. De plus, cette instance est de taille $O(|A| \times k)$ (et créer en ce temps là), qui est polynomial si on se limite aux instances non triviales de coloration, où donc $k < |A|$.

	

\end{proof}

	
\begin{definition}
	Une instance de SOUS-SOMME est une famille de nombre $\left(s_{i}\right)_{i \in  \llbracket1, n \rrbracket} \in \mathbb N$ et une cible $K \in \mathbb N$. On dit qu'une instance est positive si $$\exists I \subset   \llbracket1, k \rrbracket \: : \: \sum\limits_{i \in I} s_i = K$$
\end{definition}

\begin{rem}
	A priori, ce problème est difficile dès que $n$ devient grand, car on a pour $I$, $2^n$ choix.
\end{rem}

\begin{theorem}
	Il existe une transformation polynomiale d'une instance de SOUS-SOMME vers SAT.
\end{theorem}

\begin{proof}
	
	\paragraph{Notation}
	\begin{itemize}
		\item On identifiera ici $V$ avec $1$ et $F$ avec $0$.
		\item Notons également $M = \max \left( \max\left\{ \lceil \log s_i \rceil \middle/ i \in  \llbracket1, n \rrbracket \right\}, \: \lceil \log K \rceil \right)$.
		\item De plus, on notera $\overline{x} = \left(x_i\right)_{i \in  \llbracket0, M \rrbracket} \in \left\{ 0,1\right\}^{M+1}$.
		\item Notons $b_{i,j} \in \{0,1\}$ le $j$-ième bit de $s_i$ et $b_{K, j}$ le $j$-ième bit de $K$.
	\end{itemize}
	
	\paragraph{La somme} On cherche tout d'abord à créer une formule pour la somme. On cherche donc $F(\overline x, \overline y, \overline z , \overline r)$ tel que $F$ soit vrai si et seulement si $\sum\limits_{i = 0}^M x_i2^i + \sum\limits_{i=0}^M y_i 2^i \: = \: \sum\limits_{i=0}^M z_i 2^i$ avec $r_i$ la retenue de la $i$-ème addition.
	
	Pour cela on crée
	$$
	\begin{array}{rll}
		F\left(\overline x, \overline y, \overline z, \overline r\right) = & \neg r_{-1} & \text{(car la retenue d'entrée est 0)}\\
		\wedge & \bigwedge\limits_{i = 0}^{M} \left( x_i \oplus y_i \oplus r_{i-1} \right) = z_i & \text{(on ajoute les deux bits et la retenue précedente)} \\
		\wedge & \bigwedge\limits_{i=0}^{M} r_i = \left( \left(x_i \wedge y_i\right) \vee \left(x_i \wedge r_{i-1}\right) \vee \left(r_{i-1} \wedge y_i\right)\right) & \text{(Il y a une retenue si au moins deux bits étaient 1)}\\
		\wedge & \neg r_M & \text{(Il ne faut pas de dépassement de capacité)}\\
	\end{array}
	$$
	
	\paragraph{Le choix du sous-ensemble} On introduit maintenant les variables $(c_i)_{i\in  \llbracket1, n \rrbracket}$ dont on voudra qu'elle représente $i \in I$.
	
	\paragraph{Notre instance de SAT} On crée alors l'instance 
	$$\begin{array}{rll}
		I_{SAT} = & F\left( \left( b_{1,j} \times c_1 \right)_{j \in  \llbracket0, M \rrbracket}, \: \left( b_{2,j} \times c_2 \right)_{j \in  \llbracket0, M \rrbracket}, \, \overline{h_2}, \overline{r_2} \right) & \\
		\wedge & \bigwedge\limits_{i = 3}^{n-1} F\left( \overline{h_{i-1}}, \left( b_{i,j} \times c_i \right)_{j \in  \llbracket0, M \rrbracket}, \overline{h_i}, \overline{r_i} \right) & \\
		\wedge & F\left( \overline{h_{n-1}}, \left( b_{M,j} \times c_M \right)_{j \in  \llbracket0, M \rrbracket}, \left( b_{K,j}  \right)_{j \in  \llbracket0, M \rrbracket}, \overline{r_M} \right) &
	\end{array}
	$$
	où les $h_{i,j}$ et les $r_{i,j}$ sont des variables fraîches
	
	\begin{com}
		Là il faut expliquer à l'oral pourquoi ca marche, dans les deux sens. On peut ensuite discuter du fait qu'elle n'est pas en FNC, puisque F n'y est pas, mais presque puisque on peut remplacer les formules sur moins de 6 variables par des formules en FNC en faisant la table de vérité.
		
		On peut enfin discuter de la taille de l'instance, et pourquoi elle est linéaire en la taille de l'entrée.
	\end{com}

\end{proof}

