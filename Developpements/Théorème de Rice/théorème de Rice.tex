
\dev{Marin Malory}{\cite{Carton}}

\textit{Ce développement présente le théorème de Rice avec quelques applications simples du théorème. Puisque ce résultat permet de montrer l'indécidabilité de certains langages, il s'insère naturellement dans la leçon \ref{L27}. De manière indirecte, ce théorème donne un résultat d'indécidabilité sur la correction des programmes et peut donc illustrer la leçon \ref{L1}.}

\begin{theorem}[Théorème de Rice]~

Soit $P$ une propriété non triviale sur les langages récursivement énumérables. Le problème de savoir si le langage $L(M)$ d'une machine de Turing $M$ vérifie $P$ est indécidable.
\end{theorem}

De manière plus formelle, on a $P\subset \{ L(M) | M \text{ est une machine de Turing}\}$, et on définit $L_P$ le langage des machines de Turing vérifiant $P$ : $L_P \subset \{<M> | L(M) \in P\}$. $P$ est non triviale s'il existe $M_1$ et $M_2$ avec $<M_1>\in L_P$ et $<M_2 \notin L_P$.

\begin{example}
On peut prendre $P$ la propriété : \og le langage est non vide \fg{}, et alors $L_P =\{<M> | L(M)\neq \emptyset \}$.
\end{example}

\begin{proof}
On veut montrer que $L_P$ est indécidable. On donne une réduction de $L_\in$ à $L_p$.

Quitte à remplacer $P$ par sa négation, on suppose que le langage vide ne vérifie pas $P$. Puisque $P$ n'est pas triviale, il existe une machine $M_0$ telle que $<M_0>\in L_P$.


Pour toute paire $(M,w)$ où $M$ est une machine de Turing et $w$ un mot, on définit la machine de Turing suivante :\newline


\fbox{\begin{minipage}{0.9\linewidth}
\noindent $\mathbf{M_w=}$ \og Sur l'entrée $u$ :
\begin{enumerate}
\item Si $M$ accepte $w$, simuler $M_0$ sur $u$ et \textbf{retourner le résultat}.
\item Sinon, \textbf{rejeter}.\fg
\end{enumerate}
\end{minipage}}\newline

On veut montrer :
$$
w\in L(M) \Leftrightarrow <M_w>\in L_P
$$

Si $M$ accepte $w$, alors $L(M_w)=L(M_0)$ et donc $<M_w> \in L_P$. Réciproquement, si $M$ n'accepte pas $w$. Il y a deux cas : 
\begin{itemize}
\item Si $M$ ne s'arrête pas sur $w$, alors $M$ ne s'arrête jamais et donc $L(M)=\emptyset$
\item Si $M$ rejette $w$, $M_w$ rejette tous les mots $u$, d'où $L(M_w)=\emptyset$
\end{itemize}
Ainsi, $<M_w> \notin L_P$ puisque $\emptyset \notin L_P$.



Or, la fonction $f$ qui à $<M,w>$ associe $<M_w>$ est calculable, et donc $L_\in$ se réduit à $L_P$. Ainsi, $L_P$ est indécidable.
\end{proof}

\begin{corollary}
Le langage $L_\emptyset$ est indécidable.
\end{corollary}

\paragraph{Condition d'utilisation.} On considère un problème légèrement différent du problème précédent :
$$
L_{\neq} =\{<M,M> | L(M) \neq L(M')\}
$$
Attention, le théorème de Rice n'est pas directement applicable ici car la propriété ne porte pas sur les langages récursivement énumérables (ici sur deux langages récursivement énumérables).

On montre la proposition suivante par réduction de $L_{\emptyset}$ à $L_{\neq}$.

\begin{proposition}
$L_{\neq}$ est indécidable.
\end{proposition}

\begin{proof}
On fixe une machine $M_{\emptyset}$ qui ignore son entrée et rejette directement. On a alors $L(M_\emptyset)=\emptyset$. On pose $g$ la fonction qui à $<M>$ associe $<M,M_\emptyset>$ (qui est calculable), et on a 
$$
L(M) \neq \emptyset \Leftrightarrow L(M) \neq L(M_\emptyset)
$$
Ainsi, $g$ est une réduction de $L_{\emptyset}$ à $L_{\neq}$, donc $L_{\neq}$ est indécidable.
\end{proof}

