\dev{Marin Malory}{\cite{Pinchinat}}

\textit{
Ce développement présente une preuve d'un sens du théorème de Codd conjonctif : on montre que toute requête de l'algèbre SPC satisfaisable, il existe une formule du calcul conjonctif correspondante. Il s'insère ainsi dans la leçon \ref{L22}.
}

\paragraph{Notation.} Pour différencier les deux sémantiques, on mettra un indice $A$ pour l'algèbre relationnelle et $C$ pour le calcul conjonctif. On se permettra aussi, dans les démonstrations, d'enlever l'indice $\mathcal{M}$ pour les sémantiques.


\begin{proposition}
Pour toute requête $q$ de l'algèbre SPC satisfaisable, il existe une formule $\varphi_q$ du calcul conjonctif telle que, pour toute base de donnée $\mathcal{M}$, 
$$
\llbracket q \rrbracket^{\mathcal{M}}_A = \llbracket \varphi_q \rrbracket^{\mathcal{M}}_C
$$
\end{proposition} 


\begin{proof}
Soit $q$ une requête de l'algèbre SPC satisfaisable. On raisonne par induction sur la structure de $q$.

\begin{description}
\item[Tuple] Si $q=\langle j : d\rangle$ avec $j$ un attribut et $d$ une constante.
$$
\langle j : d \rangle \rightsquigarrow x_j = d
$$
On a alors directement 
$$
\llbracket (x_j=d)\rrbracket_C = \{\langle j : d\rangle \} = \llbracket \langle j : d\rangle \rrbracket_A 
$$

\item[Relation] Si $q=R$ :
$$
R \rightsquigarrow R(x_1,...,x_n)
$$
où $R$ possède $n$ attributs. On a
$$
\llbracket R(x_1,...,x_n) \rrbracket_C =R^\mathcal{M} = \llbracket R \rrbracket_A
$$

\item[Projection] Si $q=\pi_{j_1,...,j_n}(q')$ où $j_1,...,j_n$ sont des attributs.
$$
\pi_{j_1,...,j_n}(q') \rightsquigarrow \exists y_1,...,y_m \varphi_{q'}
$$
où par hypothèse d'induction, $\llbracket q' \rrbracket^{\mathcal{M}}_A = \llbracket \varphi_{q'} \rrbracket^{\mathcal{M}}_C$, et 
$$
\{y_1,...,y_m\} = \text{Libre}(\varphi_{q'}) \backslash \{x_{j_1},...,x_{j_n}\}
$$
Ainsi, les variables libres de $\exists y_1,...,y_m \varphi_{q'}$ sont exactement $\{x_{j_1},...,x_{j_k}\}$. Finalement 
$$
\llbracket \exists y_1,...,y_m ~ \varphi_{q'} \rrbracket_C = \pi_{j_1,...,j_k}(\llbracket \varphi_{q'} \rrbracket_C)
$$
ce qui permet de conclure.

\item[Produit cartésien] Si $q=q_1 \times q_2$. on prend 
$$
q_1 \times q_2 \rightsquigarrow \varphi_{q_1}(x_1,..,x_n) \wedge \varphi_{q_2}(y_1,...,y_m)
$$
où les $x_i$ et $y_j$ sont deux à deux disjoints.

On a alors directement 
\begin{align*}
\llbracket q_1 \times q_2 \rrbracket_A &= \{ (d_1,d_2) | d_1\in \llbracket q_1 \rrbracket_A, d_2\in \llbracket q_2 \rrbracket_A \}\\
&= \{ (d_1,d_2) | d_1\in \llbracket \varphi_{q_1} \rrbracket_C, d_2\in \llbracket \varphi_{q_2} \rrbracket_C \}\\
&= \llbracket \varphi_{q_1} \rrbracket_C \underbrace{\Join_{(i,j) |x_i=y_j}}_{\text{Join sur l'ensemble vide = produit cartésien}}\llbracket \varphi_{q_2} \rrbracket_C \\
&= \llbracket \varphi_{q_1}(x_1,...,x_n) \wedge \varphi_{q_2}(y_1,...,y_m) \rrbracket_C
\end{align*}

\item[Sélection constante] Si $q=\sigma_{j=a}(q')$,
$$
\sigma_{j=a}(q') \rightsquigarrow \varphi_{q'} \wedge (x_j=a)
$$ 
où $x_j$ est la variable associée à l'attribut $j$ et $a$ une constante. $x_j$ apparaît bien dans $\varphi_{q'}$, car sinon $j$ ne serait pas un attribut de $q'$ et on ne pourrait pas faire de sélection. En supposant que les tuples de $\llbracket q' \rrbracket_A$ ont $n$ attributs, on a:
D'un côté,
$$
\llbracket \sigma_{j=a}(q') \rrbracket_A = \{(d_1,...,d_n)\in \llbracket q' \rrbracket_A | d_j= a \} \\
$$ 
et de l'autre, en sachant que $\llbracket (x_j=a) \rrbracket_C =\{\langle a \rangle\}$ :
\begin{align*}
&\llbracket \varphi_{q'}(x_1,...,x_n) \wedge (x_j=a) \rrbracket_C \\
&= 
\llbracket \varphi_{q'}(x_1,...x_n \rrbracket_C \Join_{(j,j)} \llbracket x_j = a \rrbracket_C \\
&= \pi_{x_1,...,x_n}\{ \langle d_1,...,d_n,d_1'\rangle | (d_1,...,d_n) \in \llbracket \varphi_{q'}(x_1,...x_n) \rrbracket_C, (d_j')\in \llbracket x_j = a \rrbracket_C , d_j=d_j'\} \\
&=  \{(d_1,...,d_n)\in \llbracket q' \rrbracket_A | d_j= a \}
\end{align*}

\item[Sélection variable] Si $q=\sigma_{j=k}(q')$, le raisonnement est similaire avec :
$$
\sigma_{j=k}(q') \rightsquigarrow \varphi_{q'} \wedge (x_j =x_k)
$$
\end{description}
\end{proof}
