\dev{Emile Martinez}{Daphné Kany}

\textit{On estime que la définition des ARN est déjà admise}.

\paragraph{Insertion dans un ABR} 
\begin{enumerate}
	\item On colorie la racine en noir
	\item On insert le noeud en tant que feuille comme dans un ABR \\
	On le colore en rouge
	\item On rétablit la propriété (ii) par rotations successives en préservant la hauteur noire.
\end{enumerate}

\begin{definition}Rotation\\ 
	\begin{tabular}{ccc}
		\multirow{5}{*}{
		\begin{tikzpicture}[-, node distance=1cm]
			\node[state, scale=0.5] (q0) {$k_3$};
			\node[state, below left of = q0, scale=0.5] (q1) {$k_2$};
			\node[below left of =q1] (q2) {$t_1$};
			\node[below right of = q1] (q3) {$t_2$};
			\node[below right of = q0] (q4) {$t_3$};
			\draw (q0) edge[] node{} (q1) ;
			\draw (q1) edge[] node{} (q2) ;
			\draw (q1) edge[] node{} (q3) ;
			\draw (q0) edge[] node{} (q4) ;
		\end{tikzpicture}} &  &
		\multirow{5}{*}{\begin{tikzpicture}[-, node distance=1cm]
			\node[state, scale=0.5] (q0) {$k_2$};
			\node[state, below right of = q0, scale=0.5] (q1) {$k_3$};
			\node[below left of =q0] (q2) {$t_1$};
			\node[below left of = q1] (q3) {$t_2$};
			\node[below right of = q1] (q4) {$t_3$};
			\draw (q0) edge[] node{} (q1) ;
			\draw (q0) edge[] node{} (q2) ;
			\draw (q1) edge[] node{} (q3) ;
			\draw (q1) edge[] node{} (q4) ;
		\end{tikzpicture}} \\ & $\overset{\text{rotation droite }(k_3)}{\longrightarrow}$ & \\  & $\underset{\text{rotation gauche } (k_2)}{\longleftarrow} $ &
	\end{tabular}
\end{definition}
\begin{rem}
	La rotation préserve la structure d'ABR
\end{rem}

\begin{com}
	Expliquer la remarque à l'oral sur le dessin de la définition
\end{com}

\paragraph{Etape 3 de l'insertion}
4 cas à considérer :
%\begin{enumerate}
%	\item \raisebox{-0.5\height}{\begin{tikzpicture}[-, node distance=1cm]
%		\node[state, black, scale=0.5] (q0) {$k_3$};
%		\node[state, below left of = q0, red, scale=0.5] (q1) {$k_2$};
%		\node[state, below left of = q1, red, scale=0.5] (q2) {$x$};
%		\node[below left of =q2] (q3) {$t_0$};
%		\node[below right of = q2] (q4) {$t_1$};
%		\node[below right of = q1] (q5) {$t_2$};
%		\node[below right of = q0] (q6) {$t_3$};
%		
%		
%		\draw (q0) edge[] node{} (q1) ;
%		\draw (q1) edge[] node{} (q2) ;
%		\draw (q2) edge[] node{} (q3) ;
%		\draw (q2) edge[] node{} (q4) ;
%		\draw (q1) edge[] node{} (q5) ;
%		\draw (q0) edge[] node{} (q6) ;
%	\end{tikzpicture}}
%
%	\item \raisebox{-0.5\height}{\begin{tikzpicture}
%		\node[state, black, scale=0.5] (q0) {$k_3$};
%		\node[state, below left of = q0, red, scale=0.5] (q1) {$k_2$};
%		\node[state, below right of = q1, red, scale=0.5] (q2) {$x$};
%		\node[below left of =q2] (q3) {$t_0$};
%		\node[below right of = q2] (q4) {$t_1$};
%		\node[below left of = q1] (q5) {$t_2$};
%		\node[below right of = q0] (q6) {$t_3$};
%		
%		
%		\draw (q0) edge[] node{} (q1) ;
%		\draw (q1) edge[] node{} (q2) ;
%		\draw (q2) edge[] node{} (q3) ;
%		\draw (q2) edge[] node{} (q4) ;
%		\draw (q1) edge[] node{} (q5) ;
%		\draw (q0) edge[] node{} (q6) ;
%	\end{tikzpicture}}
%
%	\item Pareil
%	\item Pareil\\
%\end{enumerate}

\begin{enumerate}
	\item \raisebox{-0.5\height}{\begin{tikzpicture}[-, node distance=1cm]
		\node[state, black, scale=0.5] (q0) {$k_3$};
		\node[state, below left of = q0, red, scale=0.5] (q1) {$k_2$};
		\node[state, below left of = q1, red, scale=0.5] (q2) {$x$};
		\node[below left of =q2] (q3) {$t_0$};
		\node[below right of = q2] (q4) {$t_1$};
		\node[below right of = q1] (q5) {$t_2$};
		\node[below right of = q0] (q6) {$t_3$};
		
		
		\draw (q0) edge[] node{} (q1) ;
		\draw (q1) edge[] node{} (q2) ;
		\draw (q2) edge[] node{} (q3) ;
		\draw (q2) edge[] node{} (q4) ;
		\draw (q1) edge[] node{} (q5) ;
		\draw (q0) edge[] node{} (q6) ;
	\end{tikzpicture}} $\overset{\text{Rotation droite } (k_3)}{\longrightarrow} $ \raisebox{-0.5\height}{\begin{tikzpicture}[-, node distance=1cm]
		\node[state, red, scale=0.5] (q0) {$k_2$};
		\node[state, below right = 0.5 cm and 0.8 cm of q0, black, scale=0.5] (q1) {$k_3$};
		\node[state, below left = 0.5 cm and 0.8 cm of q0, black, scale=0.5] (q2) {$x$};
		\node[below left of =q2] (q3) {$t_0$};
		\node[below right of = q2] (q4) {$t_1$};
		\node[below left of = q1] (q5) {$t_2$};
		\node[below right of = q1] (q6) {$t_3$};
		
		\draw (q0) edge[] node{} (q1) ;
		\draw (q0) edge[] node{} (q2) ;
		\draw (q2) edge[] node{} (q3) ;
		\draw (q2) edge[] node{} (q4) ;
		\draw (q1) edge[] node{} (q5) ;
		\draw (q1) edge[] node{} (q6) ;
	\end{tikzpicture}}
	
	\item \raisebox{-0.5\height}{\begin{tikzpicture}
		\node[state, black, scale=0.5] (q0) {$k_3$};
		\node[state, below left of = q0, red, scale=0.5] (q1) {$k_2$};
		\node[state, below right of = q1, red, scale=0.5] (q2) {$x$};
		\node[below left of =q2] (q3) {$t_0$};
		\node[below right of = q2] (q4) {$t_1$};
		\node[below left of = q1] (q5) {$t_2$};
		\node[below right of = q0] (q6) {$t_3$};
		
		
		\draw (q0) edge[] node{} (q1) ;
		\draw (q1) edge[] node{} (q2) ;
		\draw (q2) edge[] node{} (q3) ;
		\draw (q2) edge[] node{} (q4) ;
		\draw (q1) edge[] node{} (q5) ;
		\draw (q0) edge[] node{} (q6) ;
	\end{tikzpicture}} $\overset{\text{Rotation gauche }(k_2)}{\longrightarrow}$ \raisebox{-0.5\height}{\begin{tikzpicture}[-, node distance=1cm]
	\node[state, black, scale=0.5] (q0) {$k_3$};
	\node[state, below left of = q0, red, scale=0.5] (q1) {$x$};
	\node[state, below left of = q1, red, scale=0.5] (q2) {$k_2$};
	\node[below left of =q2] (q3) {$t_2$};
	\node[below right of = q2] (q4) {$t_0$};
	\node[below right of = q1] (q5) {$t_1$};
	\node[below right of = q0] (q6) {$t_3$};
	
	
	\draw (q0) edge[] node{} (q1) ;
	\draw (q1) edge[] node{} (q2) ;
	\draw (q2) edge[] node{} (q3) ;
	\draw (q2) edge[] node{} (q4) ;
	\draw (q1) edge[] node{} (q5) ;
	\draw (q0) edge[] node{} (q6) ;
\end{tikzpicture}} $ \overset{\text{Etape 1}}{\rightarrow}$
	\item Pareil
	\item Pareil
\end{enumerate}

\paragraph{Invariant}\begin{itemize}
	\item On a toujours au plus, en tout, 2 noeuds rouges consécutifs
	\item La hauteur noire est préservée
\end{itemize}

\begin{example}
	\raisebox{-0.5\height}{\begin{tikzpicture}[-, node distance=1cm]
		\node[state, black, scale=0.5] (q0) {1};
		\node[state, black, below left of = q0, scale=0.5] (q1) {0};
		\node[state, red, below right of = q0, scale=0.5] (q2) {3};
		\node[state, black, below left of = q2, scale=0.5] (q3) {2};
		\node[state, black, below right of = q2, scale=0.5] (q4) {4};
		\node[state, red, below right of = q4, scale=0.5] (q5) {6};
		\node[state, red, below left of = q5, scale=0.5] (q6) {5};
		
		
		\draw (q0) edge[] node{} (q1) ;
		\draw (q0) edge[] node{} (q2) ;
		\draw (q2) edge[] node{} (q3) ;
		\draw (q2) edge[] node{} (q4) ;
		\draw (q4) edge[] node{} (q5) ;
		\draw (q5) edge[] node{} (q6) ;
		
	\end{tikzpicture}} $\overset{\text{Rotation\_droite(6)}}{\longrightarrow}$
	\raisebox{-0.5\height}{\begin{tikzpicture}[-, node distance=1cm]
		\node[state, black, scale=0.5] (q0) {1};
		\node[state, black, below left of = q0, scale=0.5] (q1) {0};
		\node[state, red, below right of = q0, scale=0.5] (q2) {3};
		\node[state, black, below left of = q2, scale=0.5] (q3) {2};
		\node[state, black, below right of = q2, scale=0.5] (q4) {4};
		\node[state, red, below right of = q4, scale=0.5] (q5) {5};
		\node[state, red, below right of = q5, scale=0.5] (q6) {6};
		
		
		\draw (q0) edge[] node{} (q1) ;
		\draw (q0) edge[] node{} (q2) ;
		\draw (q2) edge[] node{} (q3) ;
		\draw (q2) edge[] node{} (q4) ;
		\draw (q4) edge[] node{} (q5) ;
		\draw (q5) edge[] node{} (q6) ;
		
	\end{tikzpicture}} \\
	$\overset{\text{Rotation\_gauche(4)}}{\longrightarrow}$  \raisebox{-0.5\height}{\begin{tikzpicture}[-, node distance=1cm]
		\node[state, black, scale=0.5] (q0) {1};
		\node[state, black, below left of = q0, scale=0.5] (q1) {0};
		\node[state, red, below right of = q0, scale=0.5] (q2) {3};
		\node[state, black, below left of = q2, scale=0.5] (q3) {2};
		\node[state, red, below right of = q2, scale=0.5] (q4) {5};
		\node[state, black, below left of = q4, scale=0.5] (q5) {4};
		\node[state, black, below right of = q4, scale=0.5] (q6) {6};
		
		
		\draw (q0) edge[] node{} (q1) ;
		\draw (q0) edge[] node{} (q2) ;
		\draw (q2) edge[] node{} (q3) ;
		\draw (q2) edge[] node{} (q4) ;
		\draw (q4) edge[] node{} (q5) ;
		\draw (q4) edge[] node{} (q6) ;
		
	\end{tikzpicture}}
	$\overset{\text{Rotation\_gauche(1)}}{\longrightarrow}$
\raisebox{-0.5\height}{\begin{tikzpicture}[-, node distance=1cm]
	\node[state, red, scale=0.5] (q0) {3};
	\node[state, black, below left = 0.5cm and 0.8cm of q0, scale=0.5] (q1) {1};
	\node[state, black, below right = 0.5cm and 0.8cm of q0, scale=0.5] (q2) {5};
	\node[state, black, below left of = q1, scale=0.5] (q3) {0};
	\node[state, black, below right of = q1, scale=0.5] (q4) {2};
	\node[state, black, below left of = q2, scale=0.5] (q5) {4};
	\node[state, black, below right of = q2, scale=0.5] (q6) {6};
	
	
	\draw (q0) edge[] node{} (q1) ;
	\draw (q0) edge[] node{} (q2) ;
	\draw (q1) edge[] node{} (q3) ;
	\draw (q1) edge[] node{} (q4) ;
	\draw (q2) edge[] node{} (q5) ;
	\draw (q2) edge[] node{} (q6) ;
	
\end{tikzpicture}}

\end{example}

\paragraph{Complexité :} \begin{enumerate}
	\item $O(h)$
	\item $O(h)$
	\item $O(1)$
\end{enumerate}


\begin{proposition}
	Soit $A$ un ARN de hauteur $h$ à $n$ noeuds, alors $h = O(\log n)$
\end{proposition}

\begin{lemma}
	Soit $A$ un ARN. Alors :\begin{enumerate}
		\item $h \leq 2 h_N $ où $h_N$ est la hauteur noire.
		\item $2^{h_N} \leq n+1$
	\end{enumerate}
\end{lemma}

\begin{proof}
	\begin{enumerate}
		\item Soit $C$ un chemin de la racine à une feuille de longueur $h$.
		
		On note $h_R(C)$ le nombre de noeuds rouges de $C$ et $h_N(C)$ le nombre de noeuds noirs de $C$.
		
		$|C| = h = h_R(C) + h_N(C)$.\\ \\
		D'après (ii) (de la def) : $h_R(C) \leq h_N(C)$
		D'après (iii) : $h_N(C) = h_N$.\\
		
		Donc, $h \leq 2 h_N$.
		
		\item Procédons par induction
		\begin{com}
			Si on a pas le temps ici on écrit simplement induction.
		\end{com}
		\begin{itemize}[label = $\star$]
			\item Si l'arbre est vide, le résultat est immédiat.
			\item Soit $\mathcal A$ un ARN dont les deux fils $\mathcal A_g$ et $\mathcal A_d$ vérifient la propriété (ce sont bien des ARN).
			Alors, \begin{itemize}
				\item Si la racine de $\mathcal A$ est rouge, $2^{h_N(\mathcal A)} = 2^{h_N(\mathcal A_g)} \leq \left| \mathcal A_g \right| +1 \leq \left| \mathcal A \right| + 1$
				\item Si la racine de $\mathcal A$ est noire, alors $2^{h_N(\mathcal A)} = 2^{h_N(\mathcal A)-1} + 2^{h_N(\mathcal A)-1} = 2^{h_N(\mathcal A_g)} + 2^{h_N(\mathcal A_d)} \underset{\text{HI}}{\leq} \underset{= \left| \mathcal A \right|}{\underbrace{\left| \mathcal A_g \right| + 1 + \left|\mathcal A_d\right|}} + 1 $
			\end{itemize}
		\end{itemize}
	\end{enumerate}
\end{proof}

Ainsi, $h_N \leq \log (n+1)$ donc $h \leq 2 \log(n+1)$. Donc h = $O(\log n)$\\

Insertion : $O(\log n)$

