\dev{Emile Martinez}{Emile Martinez, Daphné Kany, sur une idée influencé par chat GPT}

\begin{algorithm}[H]
	\Entree{Un tableau $T$ d'entiers positifs\\ $|T| = n$}
	\Sortie{$T$ trié}
	
	$N \gets max(T)$ \# $O(n)$\\
	$T2 \gets$ tableau de taille $N+1$ initialisé à $0$.\\
	\Pour{$x$ dans $T$}{
		$T2[x] \gets T2[x] + 1$\\
	}
	$indice \gets 0$\\
	\Pour{$i$ de $0$ à $N$}{
		\Pour{$j$ de $0$ à $T2[i]$}{
			$T[indice] \gets i$\\
			$indice \gets indice + 1$\\
		}
	}
	\Retour{$T$}
	\caption{Tri par comptage}
\end{algorithm}

\begin{example}
	$T = [2, 1, 5, 1, 2]$\\
	$N = 5$\\
	$T2 = [0, 2, 2, 0, 0, 1]$\\
\end{example}

\begin{rem}
	Complexité spatiale : $O(N)$\\
	Complexité temporelle : $O(n+N)$\\
	\begin{com}
		Dire que si $N$ vaut 500 000 000 on l'a dans le 
	\end{com}
	Ce tri n'est pas en place.
\end{rem}


\paragraph{Problème :} Comment généraliser cet algorithme pour des tableaux contenant d'autres valeurs ?

\begin{algorithm}[H]
	\Entree{Un tableau $T$ d'entiers \textcolor{red}{/ Un tableau $T$ d'éléments de $E$} \textcolor{red}{d'éléments de $E$} \\ \textcolor{red}{Une fonction $f : E \to \N$}}
	\Sortie{$T$ trié}
	$d \gets$ dictionnaire vide\\
	\Pour{$x$ dans T}{
		$i \gets$ $recherche(d, x)$ \textcolor{red}{/ $l \gets$ $recherche(d, f(x))$}\\
		\eSi{$l \neq None$}{
			\textcolor{red}{Ajouter $x$ à $l$}\\
			$Insertion(d, x, i+1)$ \textcolor{red}{/ $Insertion(d, f(x), l)$}\\
		}{
			$Insertion(d, x, 1)$ \textcolor{red}{/ $Insertion(d, f(x), [x])$}\\
		}
	}
	$T2 = []$\\
	\Pour{$k$ dans $d$}{
		Ajouter $k$ à $T2$\\
	}
	Trier $T2$\\
	$indice \gets 0$\\
	\Pour{$i$ dans $T_2$}{
		$k \gets recherche(d, i)$ \textcolor{red}{/ $l \gets recherche(d, i)$}\\
		\Pour{$j$ de $0$ à $k-1$ \textcolor{red}{/ $x \in l$}}{
			$T[indice] \gets i$ \textcolor{red}{/ $T[indice] \gets x$}\\
			$indice \gets indice + 1$
		}
	}
	\caption{Généralisation du tri par comptage}
\end{algorithm}



\paragraph{Complexité :} \enspace \newline \begin{tabular}{rl}
	Spatiale :& $O(n)$\\
	Temporelle :& $O(n \times (C_i + C_r) + C_{tri}(m))$ en notant $m = |T2|$.\\
	Implémentation en ARN : & $O(n \log m + C_{tri}(m)) = O(n \log m + m \log m) = O(n \log m)$ \\
	Implémentation en table de hachage :& $\left\{ \begin{array}{ll}
		\text{pire cas} & O(n \times m + C_{tri}(m))\\
		\text{cas moyen} & O(n + C_{tri}(m)) = O(n + m \log m)
	\end{array}\right.$
\end{tabular}\\

Avec des tables de hachage, on obtient en moyenne sur les insertions, dans le pire cas (où $n = m$) une complexité en $O(n) + C_{tri}(n) \sim C_{tri}(n)$ (car les meilleurs algorithmes de tri par comparaison sont au pire en $\Omega(n \log n)$). Ainsi, dans le pire cas, on a asymptotiquement le même nombre de comparaison.


\paragraph{Remarque} Ce tri ne concerne que des entiers. On peut passer à n'importe quelle structure que l'on compare à travers une fonction entière par \textcolor{red}{les modifications en rouge}.

\begin{example}
	$T = ['abc', 'hello', 'world', 'aa', 'bc']$\\
	$f$ : fonction qui compte les caractères \\
	$d :\: \begin{array}{l}
	3 : \: ['abc'] \\
	5 : ['hello ', 'world']\\
	2 : ['aa']
	\end{array}$
\end{example}

\begin{proposition}
	On obtient alors un tri stable (avec une complexité spatiale en $O(|T|)$)
\end{proposition}

\begin{com}
	On connaît des algorithmes de tri par comparaison qui dans leur meilleurs cas sont linéaires (le Tim Sort de python, quand la liste est déjà trié), et dans le pire en $O(n \log n)$. Notre algorithme n'est donc pas pertinent dans toutes les situations . Il l'est si l'on veut trier des éléments avec beaucoup de redondances (ex : les français par code postaux).
\end{com}

\begin{com}
	Cette remarque peut éventuellement être écrite si on manque de temps
\end{com}

\begin{com}
	A ne faire que si vraiment on a trop trop de temps :\\
	On peut également utiliser cet algorithme pour gagner des constantes. En effet, si on connait $f:E -> \N$ croissante où $\left| f^{-1}(i) \cap T \right| = \sqrt n$, alors on peut partitionner sur ces classes, trier ces classes, puis triés à l'intérieur de ces classes. (on obtient alors une complexité en $O(n) + \sqrt n\log \sqrt n \: + \: \sqrt n \times \sqrt n \log \sqrt n \: + \: \sqrt n \times o(\sqrt n \log \sqrt n) = \dfrac{1}{2} n\log n + o(n \log n)$)
\end{com}